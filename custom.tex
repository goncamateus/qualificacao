
% ----------------------------------------------------------
% DADOS DO TRABALHO - CAPA e FOLHA DE ROSTO
% Configure os dados do trabalho aqui
% ----------------------------------------------------------
\titulo{\textbf{Divide and conquer:} A study of stratifying reward signals on Reinforcement Learning}
\autor{Mateus Gonçalves Machado}
\local{Recife}
\data{\Year}
\areaconcentracao{\textbf{Área de Concentração}: Ciências da Computação}
\orientador{\textbf{Orientador (a)}: Hansenclever de França Bassani}
\coorientador{\textbf{Coorientador (a)}: Adiel Teixeira de Almeida Filho}

\instituicao{UNIVERSIDADE FEDERAL DE PERNAMBUCO \\ CENTRO DE INFORMÁTICA \\PROGRAMA DE PÓS-GRADUAÇÃO EM CIÊNCIAS DA COMPUTAÇÃO}
\departamento{Centro de Informática}
\programa{Pós-graduação em Ciências da Computação}
\emailprograma{mgm4@cin.ufpe.br}
\siteprograma{http://cin.ufpe.br/\textasciitilde posgraduacao}

\tipotrabalho{Qualificação de Doutorado}
% O preambulo deve conter o tipo do trabalho, o objetivo, 
% o nome da instituição e a área de concentração 
%\preambulo{Trabalho apresentado ao Programa de Pós-graduação em Ciência da Computação do Centro de Informática da Universidade Federal de Pernambuco, como requisito parcial para obtenção do grau de Mestre Profissional em Ciência da Computação.}

%\preambuloatadefesa{Dissertação apresentada ao Programa de Pós-Graduação Profissional em Ciência da Computação da Universidade Federal de Pernambuco, como requisito parcial para a obtenção do título de Mestre Profissional em 04 de setembro de 2020.}

\preambulo{Trabalho apresentado ao Programa de Pós-graduação em Ciência da Computação do Centro de Informática da Universidade Federal de Pernambuco, como requisito parcial para obtenção do grau de Doutor em Ciência da Computação.}

% \preambuloatadefesa{Texto texto texto texto texto texto texto texto texto texto texto texto texto texto texto texto texto texto texto texto texto texto texto texto texto texto texto texto texto texto texto texto.}




\input{userlists}



