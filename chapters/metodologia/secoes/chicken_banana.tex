\section{Analysis of the ``Chicken-Banana'' Problem}
\label{sec:chickenbanana}

To demonstrate the efficacy of our proposed method, we introduce a novel Grid-World environment termed the ``Chicken-Banana'' problem. The environment layout is illustrated in Figure \ref{fig:chicken_banana_env} and is inspired in the allegory that chickens are much harder to capture than bananas, illustrating skills with different levels of difficulty to be learned present at the same environment. This domain is formulated as a Markov Decision Process (MDP) defined by discrete state and action spaces. The action space $\mathcal{A}$ consists of four discrete movement primitives: $\mathcal{A} = \{\text{Up, Down, Left, Right}\}$. The state space $\mathcal{S}$ is defined by the combination of the agent's spatial position and its current inventory. With 16 accessible grid positions and 4 inventory states (carrying neither object, Banana only, Chicken only, or both), the total state space size is $|\mathcal{S}| = 16 \times 4 = 64$.

\begin{figure}[ht]
    \centering
    \caption{Representation of the ``Chicken-Banana'' Grid-World environment. The agent (blue square) starts at the bottom. The environment contains three reward sources: the Chicken ($C$), the Banana ($B$), and the Gate ($G$). Note that the episode terminates upon reaching the Gate, and the path lengths differ, with $B$ being the closest and $C$ being the furthest.}
    \includegraphics[width=.48\linewidth]{images/metodologia/chicken-banana.png}
    \par\medskip\ABNTEXfontereduzida\selectfont\textbf{Source:} Author
    \label{fig:chicken_banana_env}
\end{figure}

The reward structure is composed of three distinct components: the Banana ($B$), the Chicken ($C$), and the Gate ($G$). The specific reward values and their bounds $[R_{\min}, R_{\max}]$ are defined as follows:
\begin{itemize}
    \item \textbf{Banana:} Yields a reward of $+30$ [$R_{min}=0$, $R_{\max}=30$].
    \item \textbf{Chicken:} Yields a reward of $+70$ [$R_{min}=0$, $R_{\max}=70$].
    \item \textbf{Gate:} Yields a reward of $+100$ [$R_{min}=0$, $R_{\max}=100$].
\end{itemize}
While reaching the Chicken or Banana provides a scalar reward and updates the agent's state (inventory), reaching the Gate yields the reward and triggers immediate \textbf{episode termination}.

Notably, the objects are arranged by increasing distance from the starting position in the following order: Banana, Gate, and Chicken. This configuration is designed to demonstrate that a standard Q-Learning agent may fail to discover the Chicken ($+70$) if exploration strategies do not explicitly encourage visiting more distant states before the episode terminates at the Gate. This highlights the issue of premature termination, where an agent may finish the episode without collecting all available rewards.

From an exploration perspective, the Banana represents the most accessible component, followed by the Gate, and finally the Chicken. However, within the \gls{dylam} framework, we eschew terms such as ``easiest'' or ``hardest.'' Instead, we characterize a component as sub-optimal or unoptimized relative to its specific maximum reward vector, $\vec{R}_{\max}$. If stochastic exploration allows the agent to reach the most distant component, \gls{dylam} adapts to optimize the previously neglected components. In the following section, we provide an analysis of how \gls{dylam} modulates the agent's behavior to satisfy the distinct requirements of each reward component throughout the training.

\subsection{Reward Components Development}

As previously discussed, this environment was designed to explicitly expose how each reward component is prioritized throughout the training process. It is important to emphasize that an agent’s learning dynamics do not necessarily align with human intuition. In this experimental setup, the reward weight vector $\vec{R}$ was allowed to vary from $\vec{0}$ up to $\vec{R}_{\max}=\{30, 70, 100\}$. The DyLam replay buffer size was set to $E=10$, with a smoothing factor of $\tau_\lambda=0.995$. Exploration followed an $\epsilon$-greedy decay strategy, with $\epsilon$ decreasing from $1$ to $0.05$ using a per-episode decay factor of $0.9988$. The performance of the proposed methods is compared against baseline approaches in \fref{fig:meth/rew_chicken_banana}.

\noindent
\begin{figure}[ht]
\caption{Reward components development and cumulative reward performance comparison on Chicken-Banana environment during 2000 training episodes. The cumulative episode reward of each component is normalized according to $\vec{R}_{max}=\{30,70,100\}$. The results are the mean of 10 random seeds and using $\epsilon$-greedy decay algorithm. Solid lines represent the mean performance of each method, while the shaded regions indicate the minimum–maximum range across seeds. Comparisons are shown for Q-Learning (baseline), Q-Decomp (baseline), UDC, and DyLam.}
\centering
\begin{subfigure}[b]{.49\textwidth}
    \centering
    \caption{Banana component training development.}
    \includegraphics[width=\textwidth]{images/metodologia/res/reward-banana.pdf}
    \label{fig:meth/rew_chicken_banana-a}
\end{subfigure}
\begin{subfigure}[b]{.49\textwidth}
    \centering
    \caption{Chicken component training development.}
    \includegraphics[width=\textwidth]{images/metodologia/res/reward-chicken.pdf}
    \label{fig:meth/rew_chicken_banana-b}
\end{subfigure}
\begin{subfigure}[b]{.49\textwidth}
    \centering
    \caption{Gate component training development.}
    \includegraphics[width=\textwidth]{images/metodologia/res/reward-gate.pdf}
    \label{fig:meth/rew_chicken_banana-c}
\end{subfigure}
\begin{subfigure}[b]{.49\textwidth}
    \centering
    \caption{Cumulative episode reward.}
    \includegraphics[width=\textwidth]{images/metodologia/res/reward-total.pdf}
    \label{fig:meth/rew_chicken_banana-d}
\end{subfigure}
\par\medskip\ABNTEXfontereduzida\selectfont\textbf{Source:} Author
\label{fig:meth/rew_chicken_banana}
\end{figure}

Notably, DyLam is the only method that consistently learns to collect all objects before terminating the episode. Although no explicit time-penalty component is defined, the remaining methods fail to identify trajectories that include the Chicken. Once the agent reaches the Chicken, these methods are unable to subsequently reach the Gate objective, resulting in a lower cumulative episode reward and making the Chicken collection behavior unattractive under their respective learning dynamics.

When comparing UDC to the static-weight baselines, an early emergence of the Banana component is observed under UDC that does not occur in standard Q-Learning. This difference follows directly from how bootstrap targets are constructed. In standard decomposed Q-learning, each component is updated independently according to Equation \ref{eq:zimdars_original}:
\[
Q_i(s_t,a_t)\leftarrow r_i(s_t,a_t,s_{t+1}) + \gamma \max_{a} Q_i(s_{t+1},a),
\]
so the Banana component can only propagate value along trajectories that are optimal for Banana itself. If reaching the Banana requires temporarily sacrificing progress on other components such as the Gate or Chicken, this value signal is suppressed and fails to propagate backward.

In contrast, UDC uses the globally greedy policy $\pi_G$ to select the bootstrap action (Equation \ref{eq:realistic_component_update}), 
\[
Q_i(s_t,a_t)\leftarrow r_i(s_t,a_t,s_{t+1}) + \gamma Q_i\!\bigl(s_{t+1},\pi_G(s_{t+1})\bigr),
\]
which couples all components through a shared decision rule. As a result, even if a trajectory toward the Banana is not locally optimal for the Banana component alone, it is still reinforced whenever it remains globally optimal with respect to the sum of all components. In particular, trajectories that reach the Banana while still allowing the agent to subsequently reach the Gate receive positive backup signals. This mechanism enables sparse Banana-reaching events to accumulate value early in training, explaining the behavior observed in Figures~\ref{fig:meth/rew_chicken_banana-a} and~\ref{fig:meth/rew_chicken_banana-c}.


For the Q-Decomposition method, each reward component is learned at different stages of training; however, the agent fails to learn policies that achieve all objectives simultaneously. This limitation is likely due to the locally optimal and potentially misleading updates induced by independent component learning, which hinder coordinated multi-objective behavior.

\noindent
\begin{figure}[ht]
\caption{Cumulative Episode Reward (\textbf{CER}) and the evolution of the adaptive $\lambda$-weights for each reward component in the Chicken–Banana environment over 2000 training episodes.}
\centering
\includegraphics[width=\textwidth]{images/metodologia/res/reward-lambda-dylam.pdf}
\par\medskip\ABNTEXfontereduzida\selectfont\textbf{Source:} Author
\label{fig:meth/lambdas_chicken_banana}
\end{figure}

Figure~\fref{fig:meth/lambdas_chicken_banana} illustrates the evolution of the $\lambda$-weights throughout training for DyLam. Notably, the Chicken component receives a substantially higher prioritization due to its poor initial performance, maintaining approximately $75\%$ of the total weight. In contrast, the Banana and Gate components are learned relatively quickly—within roughly the first 200 episodes, allowing DyLam to progressively down-weight them and redirect learning capacity toward the more challenging Chicken objective.

Between approximately 250 and 1250 training episodes, DyLam exhibits less stable learning dynamics when compared to the other methods shown in \fref{fig:meth/rew_chicken_banana}. This behavior arises because the greedy policy $\pi_G(s_t)$ increasingly selects actions that prioritize reaching the Chicken. Consequently, even though the agent has already learned reliable policies for reaching the Banana and the Gate, it continues to favor trajectories that attempt to improve the underperforming Chicken component. As the frequency of successful Chicken-reaching episodes increases—particularly between 1000 and 1250 episodes—the learning process stabilizes, and the agent eventually converges to collecting all objects in every episode.

When inspecting DyLam’s update mechanism, one might initially perceive that the adaptive weight vector $\vec{\lambda}$ varies more aggressively than suggested by Theorem~\ref{theo:dyl_pg_theo}. However, this apparent discrepancy is explained by the buffer size $E=10$, which enforces that the same $\vec{\lambda}$ is applied for ten consecutive $Q$-function updates. When an update to $\vec{\lambda}$ does occur, it retains $99.5\%$ of its previous value due to the smoothing factor $\tau_\lambda=0.995$, resulting in gradual yet perceptible shifts in prioritization.

\subsection{Ablation Study}

The DyLam framework introduces several sensitive design choices and hyperparameters that can significantly influence its learning behavior. In particular, these include the update rate of the reward-weighting coefficients, denoted by $\tau_{\lambda}$, and the capacity of DyLam’s experience buffer, $\overline{RB^{i}_{E}}$. Additionally, the formulation in Equation~\ref{eq:lambda_func_fixed} employs a softmax-based normalization, which, while ensuring bounded and comparable weights, may introduce abrupt changes in the reward prioritization dynamics. 

Furthermore, as DyLam is evaluated in the Chicken--Banana environment within a value-based reinforcement learning paradigm, the choice of exploration strategy plays a crucial role in overall performance~\cite{sutton2018reinforcement}. In our experiments, we adopt an $\epsilon$-greedy exploration policy with a decaying schedule, which introduces an additional sensitive hyperparameter governing the exploration--exploitation trade-off.

In this subsection, we conduct a systematic ablation study by varying each of the aforementioned parameters independently. The goal is to isolate and quantify their individual impact on learning dynamics, convergence properties, and final performance, thereby providing a clearer understanding of how each component contributes to DyLam’s dynamic reward-weighting mechanism.

\subsubsection{DyLam update rate}

This ablation study investigates the influence of the update rate $\tau_{\lambda}$ on the temporal evolution of the reward-weighting coefficients in DyLam. By evaluating multiple values of $\tau_{\lambda}$, we aim to characterize how quickly the method reacts to variations in the reward signals and how this responsiveness affects training stability, convergence dynamics, and asymptotic performance.

Figure~\ref{fig:ablation/tau} reports the results obtained for three different values of $\tau_{\lambda}$. As expected, smaller values of $\tau_{\lambda}$ lead to more pronounced fluctuations in the reward components, reflecting a higher sensitivity of the weighting mechanism to short-term variations. As a direct consequence of this increased reward instability, the adaptive coefficients $\vec{\lambda}$ also exhibit larger oscillations as $\tau_{\lambda}$ decreases.

Despite this heightened variability, the averaged performance indicates that the agent was still able to achieve all objectives in at least half of the episodes. This observation partially challenges the quasi-static update assumption for $\vec{\lambda}$ discussed in Section~\ref{sec:dylam}. However, such behavior is unlikely to generalize to environments with higher-dimensional state–action spaces, where rapid updates of $\vec{\lambda}$ would likely exacerbate instability and hinder convergence. Therefore, the observed robustness should be interpreted as a consequence of the highly constrained nature of this toy-problem setting.

\begin{figure}[ht]
    \centering
    \caption{Evolution of the reward components (top row) and adaptive $\lambda$-weights (middle row) in the Chicken--Banana environment across 10 seeds, together with a single-run analysis of the $\lambda$-weights (bottom row), over 2000 training episodes for different update rates $\tau_{\lambda} \in \{0.5, 0.7, 0.99\}$. The corresponding curves are shown in green, red, and blue, respectively. Solid curves represent the mean across seeds, while the shaded regions indicate the minimum--maximum range, illustrating the increased variability induced by faster update rates.}
    \includegraphics[width=\linewidth]{images//metodologia//tau/combined_results.pdf}
    \par\medskip\ABNTEXfontereduzida\selectfont\textbf{Source:} Author
    \label{fig:ablation/tau}
\end{figure}

\subsubsection{DyLam experience buffer}

This ablation study examines the impact of the experience buffer capacity $E$ on DyLam’s behavior. By varying the buffer size, we analyze the trade-off between short-term responsiveness to recent interactions and the increased stability expected from averaging over a longer history of experiences.

Figure~\ref{fig:ablation/rb} reveals an interesting discrepancy between single-run and aggregated analyses. While the single-run results exhibit the expected behavior—namely, smoother adaptations of the $\vec{\lambda}$ coefficients as $E$ increases—the aggregated results across 10 seeds do not show a similarly pronounced effect. In particular, although the single-run analysis suggests that larger buffers effectively dampen rapid fluctuations in $\vec{\lambda}$, the overall variability observed across seeds remains largely unchanged when modifying $E$.

Consistent with the previous ablation study, variations in the buffer capacity have only a marginal impact on the mean evolution of the reward components in this environment. This outcome reinforces the interpretation that the Chicken--Banana task constitutes a highly constrained toy problem, in which structural properties of the environment dominate the learning dynamics and partially mask the stabilizing effects typically associated with larger experience buffers.

\begin{figure}[ht]
    \centering
    \caption{Evolution of the reward components (top row) and adaptive $\lambda$-weights (middle row) in the Chicken--Banana environment across 10 seeds, together with a single-run analysis of the $\lambda$-weights (bottom row), over 2000 training episodes for experience buffer sizes $E \in \{10, 50, 100\}$. The corresponding curves are shown in blue, green, and red, respectively. Solid curves denote the mean across seeds, while the shaded regions indicate the minimum--maximum range.}
    \includegraphics[width=\linewidth]{images//metodologia//rb/combined_results.pdf}
    \par\medskip\ABNTEXfontereduzida\selectfont\textbf{Source:} Author
    \label{fig:ablation/rb}
\end{figure}


\subsubsection{DyLam normalizer}

This ablation study investigates the normalization mechanism employed in Equation~\ref{eq:lambda_func_fixed}. In particular, we analyze the effect of the proposed softmax normalizer on the smoothness of the reward-weight updates and compare it against alternative normalization strategies, with the goal of assessing whether softer transitions can improve learning stability without sacrificing adaptivity. In addition to the softmax normalization, we consider an $\ell_1$ normalizer,
\[
\frac{\overline{RB^i_e}}{\sum_{j=1}^{N} \overline{RB^j_e}} ,
\]
and a min--max normalizer,
\[
\frac{\bigl(\max \overline{RB_e} - \overline{RB^i_e}\bigr)}{\bigl(\max \overline{RB_e} - \min \overline{RB_e}\bigr)}.
\]

Figure~\ref{fig:ablation/normalizer} summarizes the results of this comparison. As anticipated, the min--max normalization leads to noticeably noisier dynamics and inferior performance relative to the other methods, a behavior consistent with its well-known sensitivity to extreme values and rapid changes in the reward signal. In contrast, the $\ell_1$ normalizer achieves performance and $\vec{\lambda}$ adaptation patterns comparable to those obtained with the proposed softmax approach. This suggests that, while the softmax normalizer provides smoother and more stable updates in general, the $\ell_1$ normalization constitutes a viable alternative in scenarios where the use of softmax may be undesirable or impractical.

\begin{figure}[ht]
    \centering
    \caption{Evolution of the reward components (top row) and adaptive $\lambda$-weights (bottom row) in the Chicken--Banana environment across 10 seeds, over 2000 training episodes, using $\ell_1$, min--max, and softmax normalization strategies (green, red, and blue, respectively) in Equation~\ref{eq:lambda_func_fixed}. Solid curves denote the mean across seeds, while the shaded regions indicate the minimum--maximum range.}
    \includegraphics[width=\linewidth]{images//metodologia//normalizer/combined_results.pdf}
    \par\medskip\ABNTEXfontereduzida\selectfont\textbf{Source:} Author
    \label{fig:ablation/normalizer}
\end{figure}

\subsubsection{Epsilon-greedy decay}

Finally, we analyze the sensitivity of DyLam to the decay schedule of the $\epsilon$-greedy exploration policy. Different decay rates are evaluated to examine how the exploration--exploitation trade-off interacts with DyLam’s adaptive reward weighting and how this interaction influences convergence speed and final policy quality.

Figure~\ref{fig:ablation/epsilon} indicates that the $\epsilon$-decay rate is the most sensitive hyperparameter among those considered in our ablation study. When selecting an appropriate decay for DyLam, we hypothesize that effective learning requires maintaining a sufficiently high exploration level—approximately $50$--$70\%$—for each reward component. In this setting, a decay factor of $\epsilon_d = 0.99$ satisfies this criterion, and the resulting behavior shown in Figure~\ref{fig:ablation/epsilon} aligns with this expectation. In contrast, decay values of $\epsilon_d = 0.8$ and $\epsilon_d = 0.9$ lead to premature exploitation, preventing adequate exploration of all reward components.

However, when these results are contrasted with the $Q$-learning baseline in Figure~\ref{fig:meth/rew_chicken_banana}, it becomes evident that even under an exploration regime that is optimal for DyLam, the baseline method fails to learn all reward components. This observation suggests that DyLam’s hyperparameters do not constitute the primary bottleneck in performance. Instead, the limitations of the underlying learning algorithm dominate the observed behavior. In this sense, DyLam should be interpreted as a framework that facilitates curriculum learning when paired with a sufficiently expressive and capable base method, rather than as a source of restrictive inductive bias.

\begin{figure}[ht]
    \centering
    \caption{Evolution of the reward components (top row) and adaptive $\lambda$-weights (bottom row) in the Chicken--Banana environment across 10 seeds, over 2000 training episodes, for different $\epsilon$-decay values $\epsilon_d \in \{0.8, 0.9, 0.99\}$. The corresponding curves are shown in blue, green, and red, respectively. Solid curves denote the mean across seeds, while the shaded regions indicate the minimum--maximum range.}
    \includegraphics[width=\linewidth]{images//metodologia//epsilon/combined_results.pdf}
    \par\medskip\ABNTEXfontereduzida\selectfont\textbf{Source:} Author
    \label{fig:ablation/epsilon}
\end{figure}

Overall, this ablation study demonstrates that the primary ``thermometer'' hyperparameters introduced by DyLam do influence learning dynamics, but they are not the dominant factor limiting performance. Rather, the choice and capacity of the underlying reinforcement learning algorithm exert a substantially stronger effect on the agent’s ability to solve the task, with DyLam acting as an enabling mechanism rather than a constraining one.


Having established an analytical and empirical validation of DyLam and UDC on this controlled toy environment, the following chapter evaluates their performance across a diverse set of Reinforcement Learning benchmarks. These experiments demonstrate that the proposed methods scale to state-of-the-art frameworks while delivering competitive performance and interpretable learning dynamics.
