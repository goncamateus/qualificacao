\section{Neural Networks}
\label{sec:neural-networks}

\gls{nn} are computational models inspired by the structure and function of the human brain, originally proposed as simplified models of biological neurons \cite{mcculloch1943logical}. Since their formalization with the perceptron model \cite{rosenblatt1958perceptron}, \gls{nn} have evolved into powerful function approximators capable of representing complex, high-dimensional, and non-linear relationships \cite{hornik1989multilayer}. This universal approximation capability, combined with advances in optimization and hardware acceleration, has led to the widespread adoption of neural networks across machine learning applications, including vision, language, and control tasks.

In \gls{rl}, neural networks have become a key component in modern algorithms. They are typically used to approximate value functions, policies, or environment dynamics, enabling agents to handle large or continuous state and action spaces where tabular methods are infeasible \cite{mnih2015human, ddpg}. This integration marks the emergence of Deep Reinforcement Learning, where the representational capacity of deep networks is leveraged to learn directly from raw sensory data or high-dimensional observations.

This section introduces the fundamental concepts behind neural networks that are relevant for understanding their role in RL algorithms. We begin with the basic building blocks—perceptrons, feedforward networks and activation functions and progressively explore training mechanisms.
\subsection{Neuron to artificial neuron and nets}
\label{sec:perceptron}

The perceptron, introduced by \citeonline{rosenblatt1958perceptron}, is a simple computational model that mimics the behavior of a biological neuron. It computes a weighted sum of its input features, adds a bias term, and applies a transformation to produce the final output. Formally, for an input vector $\mathbf{x} \in \mathbb{R}^n$, weights $\mathbf{w} \in \mathbb{R}^n$, and bias $b \in \mathbb{R}$, the output $y$ of a perceptron is given by:

\begin{equation}
y = \phi\left(\mathbf{w}^\top \mathbf{x} + b \right),
\end{equation}

where $\phi(\cdot)$ denotes a non-linear activation function, responsible for introducing non-linearity into the model. While the original perceptron used the Heaviside step function, modern networks typically employ smoother and differentiable alternatives such as the ReLU or sigmoid functions, though the specifics are beyond the scope of this work~\cite{dubey2021activation, lederer2021activation}.

\begin{figure}[ht]
    \caption{Structure of a neuron versus the structure of a perceptron.}
    \centering
    \includegraphics[width=\textwidth]{images/fundamentacao/nn/perceptron.pdf}
    \label{fig:perceptron}
    \par\medskip\ABNTEXfontereduzida\selectfont\textbf{Source:} Author
    
\end{figure}

While a single-layer perceptron can only represent linearly separable functions, stacking multiple layers of perceptrons enables the construction of \emph{feedforward neural networks} (also known as \gls{mlp}). In a feedforward network, neurons are organized in successive layers, where each neuron in one layer receives inputs from the previous layer and sends its output to the next layer, without forming cycles.

\begin{figure}[ht]
    \caption{Basic architecture of a feedforward neural network (MLP) with input, two hidden, and output layers.}
    \centering
    \includegraphics[width=0.6\textwidth]{images/fundamentacao/nn/mlp.pdf}
    \label{fig:mlp}
    \par\medskip\ABNTEXfontereduzida\selectfont\textbf{Source:} Author
\end{figure}

This layered architecture allows the network to learn hierarchical feature representations, with each layer extracting increasingly abstract features from the input data. The universal approximation theorem \cite{hornik1989multilayer} guarantees that feedforward networks with at least one hidden layer and suitable activation functions can approximate any continuous function on a compact domain, given sufficient neurons.

Feedforward networks form the backbone of many reinforcement learning algorithms, where they are commonly used to approximate value functions, policies, or models of the environment \cite{silver2016mastering, ddpg, sac}.

\subsection{Backpropagation and Gradient Descent}
\label{sec:backpropagation}

Training a neural network involves adjusting its parameters—namely the weights and biases—to minimize a loss function that quantifies the error between the model's predictions and the expected outputs. One of the most widely used optimization strategies for this purpose is \emph{gradient descent} \cite{ruder2016overview}.

Gradient descent is an iterative algorithm that updates the parameters in the direction of the negative gradient of the loss function with respect to those parameters. Given a parameter vector $\boldsymbol{\theta}$ and a learning rate $\alpha > 0$, the update rule at iteration $t$ is defined as:

\begin{equation}
\boldsymbol{\theta}^{(t+1)} = \boldsymbol{\theta}^{(t)} - \alpha \nabla_{\boldsymbol{\theta}} \mathcal{L}(\boldsymbol{\theta}^{(t)}),
\end{equation}

where $\mathcal{L}$ is the loss function and $\nabla_{\boldsymbol{\theta}} \mathcal{L}$ denotes its gradient with respect to the parameters.

Several variants of gradient descent exist, including \gls{sgd}, mini-batch gradient descent, and adaptive methods such as Adam and RMSProp \cite{kingma2014adam}. Although more advanced optimizers are often used in practice, this work focuses on the core gradient-based optimization approach for clarity and educational purposes.

To apply gradient descent in the context of deep neural networks, it is necessary to compute the gradient of the loss function with respect to each individual weight and bias in the network. This is accomplished through the \textit{backpropagation} algorithm, introduced by \citeonline{rumelhart1986learning}.

Backpropagation efficiently computes the required gradients using the chain rule of calculus, propagating the error from the output layer backward through the network. This enables updating each parameter based on its contribution to the total error.

The training process typically follows two steps:
\begin{itemize}
  \item \textbf{Forward pass:} The input is propagated through the network to compute predictions and the corresponding loss.
  \item \textbf{Backward pass:} The gradients of the loss with respect to the network's parameters are computed layer by layer, from the output layer back to the input layer.
\end{itemize}

Algorithm~\ref{alg:backprop} outlines the general training procedure for a neural network using gradient descent and backpropagation.

\begin{algorithm}[H]
\caption{Neural Network Training with Backpropagation}
\label{alg:backprop}
\begin{algorithmic}[1]
\Require Training data $(\mathbf{x}, \mathbf{y})$, learning rate $\alpha$, number of epochs $T$
\State Initialize weights $\mathbf{W}$ and biases $\mathbf{b}$ randomly
\For{epoch $= 1$ to $T$}
    \For{each mini-batch $(\mathbf{x}_i, \mathbf{y}_i)$}
        \State \textbf{Forward pass:} compute predictions $\hat{\mathbf{y}}_i$
        \State Compute loss $\mathcal{L}(\hat{\mathbf{y}}_i, \mathbf{y}_i)$
        \State \textbf{Backward pass:} compute gradients $\nabla_{\mathbf{W}, \mathbf{b}} \mathcal{L}$
        \State Update weights: $\mathbf{W} \leftarrow \mathbf{W} - \alpha \nabla_{\mathbf{W}} \mathcal{L}$
        \State Update biases: $\mathbf{b} \leftarrow \mathbf{b} - \alpha \nabla_{\mathbf{b}} \mathcal{L}$
    \EndFor
\EndFor
\end{algorithmic}
\end{algorithm}

