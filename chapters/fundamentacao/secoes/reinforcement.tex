\section{Reinforcement Learning}
\label{sec:rl}
Reinforcement learning is the problem of learning how to map situations to actions in order to maximize a numerical reward signal. The learner, often called an \textbf{agent}, must discover which actions yield the highest reward through a process of trial and error. Importantly, these actions do not only influence the immediate reward, but also affect future situations and subsequent rewards~\cite{sutton2018reinforcement}.

This section introduces the fundaments of reinforcement learning, first with the \emph{Markov Decision Process} (MDP), which provides the mathematical foundation for modeling decision-making under uncertainty. Then, we examine two major families of learning algorithms: \emph{value-based methods}, which rely on estimating value functions to derive optimal policies, and \emph{policy gradient methods}, which directly optimize the policy through gradient-based approaches. Together, these subsections provide the theoretical and algorithmic basis necessary to understand and develop reinforcement learning solutions.

\subsection{Finite Markov Decision Process}
\label{sec:mdp}

The \gls{mdp} is a mathematical framework used to formalize sequential decision-making problems in \gls{rl}, where the agent learns from incremental feedback. It provides a well-defined structure to model the interaction between an agent and its environment. As previously introduced, the agent observes the environment, takes actions, and receives feedback in the form of rewards. This interactive process, at its core, defines a \gls{mdp}~\cite{sutton2018reinforcement}.

\begin{figure}
    \caption{Representation of a \gls{mdp}. Source: Author}
    \centering
    \includegraphics[width=0.75\linewidth]{images/fundamentacao/rl/mdp.png}
    \label{fig:fundamentacao/mdp_basic}
    \par\medskip\ABNTEXfontereduzida\selectfont\textbf{Source:} Author
\end{figure}

Formally, a finite Markov Decision Process is defined by a tuple $(\mathcal{S}, \mathcal{A}, \mathcal{P}, \mathcal{R}, \gamma)$, where:

\begin{itemize}
    \item $\mathcal{S}$ is a finite set of states.
    \item $\mathcal{A}$ is a finite set of actions.
    \item $\mathcal{P}: \mathcal{S} \times \mathcal{A} \times \mathcal{S} \rightarrow [0,1]$ is the state transition probability function, where $\mathcal{P}(s'|s,a)$ denotes the probability of transitioning to state $s'$ after taking action $a$ in state $s$.
    \item $\mathcal{R}: \mathcal{S} \times \mathcal{A} \rightarrow \mathbb{R}$ is the reward function, where $\mathcal{R}(s,a)$ denotes the expected immediate reward received after taking action $a$ in state $s$.
    \item $\gamma \in [0,1]$ is the discount factor, which determines the importance of future rewards.
\end{itemize}

At each discrete time step $t = 0, 1, 2, \dots$, the agent observes the current state $s_t \in \mathcal{S}$, selects an action $a_t \in \mathcal{A}(s_t)$, receives a scalar reward $r_{t+1} \in \mathbb{R}$, and transitions to a new state $s_{t+1} \in \mathcal{S}$ according to the transition probability $\mathcal{P}(s_{t+1} | s_t, a_t)$. This process is illustrated in Figure~\ref{fig:fundamentacao/mdp_basic}.




\subsection{Bellman Equations}
\label{sec:bellman}

The formal structure of a \gls{mdp} enables us to reason about the quality of states and actions through the concept of value functions. These functions quantify how good it is for an agent to be in a certain state, or to take a particular action in that state, under a given policy. The recursive formulation that defines these value functions is known as the \emph{Bellman equation}~\cite{sutton2018reinforcement}.

Named after Richard Bellman, who introduced dynamic programming in the 1950s \cite{bellman1966dynamic}, the Bellman equations lie at the heart of reinforcement learning. They provide a fundamental insight: the value of a state can be decomposed into the immediate reward and the value of subsequent states. This recursive structure allows learning algorithms to estimate long-term returns without explicitly exploring all future trajectories.

Given a policy $\pi: \mathcal{S} \rightarrow \mathcal{A}$ that maps states to actions, the \emph{\textbf{state-value function}}, $v_\pi(s)$ is defined as the expected return when starting in state $s$ and following $\pi$ thereafter:

\begin{equation}
v_\pi(s) = \mathbb{E}_\pi \left[ \sum_{t=0}^\infty \gamma^t r_{t+1} \,\middle|\, s_0 = s \right].
\end{equation}

This function satisfies the \emph{\textbf{Bellman expectation equation}}:

\begin{equation}
v_\pi(s) = \sum_{a \in \mathcal{A}} \pi(a|s) \sum_{s' \in \mathcal{S}} \mathcal{P}(s'|s,a) \left[ \mathcal{R}(s,a) + \gamma v_\pi(s') \right].
\label{eq:bellman_value}
\end{equation}

Similarly, the \emph{\textbf{action-value function}} $q_\pi(s,a)$ defines the expected return after taking action $a$ in state $s$ and then following policy $\pi$:

\begin{equation}
q_\pi(s,a) = \mathbb{E}_\pi \left[ \sum_{t=0}^\infty \gamma^t r_{t+1} \,\middle|\, s_0 = s, a_0 = a \right],
\end{equation}

\noindent
and satisfies the corresponding Bellman equation:

\begin{equation}
q_\pi(s,a) = \sum_{s' \in \mathcal{S}} \mathcal{P}(s'|s,a) \left[ \mathcal{R}(s,a) + \gamma \sum_{a' \in \mathcal{A}} \pi(a'|s') q_\pi(s',a') \right].
\label{eq:bellman_action_value}
\end{equation}

While the Bellman expectation equations describe the value of a policy $\pi$, the goal in reinforcement learning is often to find an \emph{optimal policy} $\pi^*$—one that maximizes the expected return from every state. To reason about this, we define the \emph{optimal value functions}:

\begin{itemize}
    \item The \textbf{optimal state-value function}:
    \begin{equation}
        v_*(s) = \max_\pi v_\pi(s),
    \end{equation}
    which gives the highest expected return achievable from state $s$ under any policy.
    
    \item The \textbf{optimal action-value function}:
    \begin{equation}
        q_*(s,a) = \max_\pi q_\pi(s,a),
    \end{equation}
    which gives the highest expected return achievable from state $s$ when taking action $a$ and following the best policy thereafter.
\end{itemize}

These functions satisfy the \emph{Bellman optimality equations}, which are nonlinear due to the presence of the max operator. The optimal state-value function obeys:

\begin{equation}
v_*(s) = \max_{a \in \mathcal{A}} \sum_{s' \in \mathcal{S}} \mathcal{P}(s'|s,a) \left[ \mathcal{R}(s,a) + \gamma v_*(s') \right].
\label{eq:bellman_opt_value}
\end{equation}

Likewise, the optimal action-value function satisfies:

\begin{equation}
q_*(s,a) = \sum_{s' \in \mathcal{S}} \mathcal{P}(s'|s,a) \left[ \mathcal{R}(s,a) + \gamma \max_{a' \in \mathcal{A}} q_*(s',a') \right].
\label{eq:bellman_opt_q}
\end{equation}

Solving these equations directly yields the optimal value functions, from which an optimal policy $\pi^*$ can be derived by acting greedily:

\begin{equation}
\pi^*(s) = \arg\max_{a \in \mathcal{A}} q_*(s,a).
\end{equation}

These equations form the theoretical foundation for many dynamic programming and reinforcement learning algorithms, which aim to approximate or iteratively compute $v_*$ or $q_*$ in order to find optimal behavior

\section{Monte Carlo Learning}
\label{sec:monte_carlo}

\gls{mc} methods are a foundational class of model-free reinforcement learning techniques that estimate value functions and optimize policies by relying exclusively on sampled experience. The term ``Monte Carlo'' refers to methods that solve problems through random sampling — a concept originating from simulations used in physics and mathematics, and named after the famous casino in Monaco due to their reliance on stochasticity and chance \cite{kroese2014monte}. In reinforcement learning, this stochastic element comes from the interaction between the agent and the environment, where the agent collects data through episodes and averages the observed returns.

Unlike dynamic programming, which requires full knowledge of the environment’s transition and reward functions, Monte Carlo methods operate without access to a model. Instead, they use complete sequences of experience — also known as episodes — to estimate the expected return from states or state-action pairs. These experiences are formed by tuples of the form $(s, a, r, s')$, representing a single interaction between the agent and the environment.

The experience collected can be either \textit{real}, obtained through direct interaction with the environment, or \textit{simulated}, produced by a learned or hand-crafted model. Regardless of the source, Monte Carlo methods assume that these episodes are representative of the agent’s behavior under a fixed policy and that they can be used to estimate long-term returns by averaging the observed outcomes.

Because Monte Carlo methods require the return to be computed over an entire episode, they are naturally suited for \textbf{episodic tasks}. In such cases, policy updates typically occur at the end of each episode. For continuing tasks, updates are often performed every fixed number of steps, simulating artificial episodes to approximate returns. This episodic nature introduces a key characteristic of \gls{mc} methods: they are incremental only in an \textit{episode-by-episode} sense, but not in a \textit{step-by-step} fashion~\cite[p.~91]{sutton2018reinforcement}.

However, this reliance on episodic returns also introduces challenges. One of them is the \textbf{nonstationarity} of the return distribution: as the policy changes over time, previously collected experiences may no longer reflect the behavior of the current policy, making naive averaging potentially biased. For this reason, care must be taken when applying \gls{mc} methods in non-stationary or off-policy settings, often requiring additional tools such as importance sampling to correct for distributional shifts \cite{sutton2018reinforcement}.

Once the agent collects complete episodes under a fixed policy $\pi$, the next step is to estimate the value of states encountered during these episodes. This process is known as \textbf{Monte Carlo prediction}, and its goal is to compute the state-value function $v_\pi(s)$ using empirical returns.

The return, denoted by $G_t$, represents the cumulative reward obtained by the agent from time step $t$ onward. For a given trajectory $(S_0, A_0, R_1, \ldots, S_T)$, the return is defined as:
\begin{equation}
    G_t = R_{t+1} + \gamma R_{t+2} + \gamma^2 R_{t+3} + \cdots = \sum_{k=0}^\infty \gamma^k R_{t+k+1}
\end{equation}
In episodic tasks, where the episode eventually terminates, this summation is finite and can be computed exactly.

To estimate $v_\pi(s)$, Monte Carlo methods average the observed returns following visits to state $s$ across multiple episodes. More formally, we use:
\[
v_\pi(s) \approx \frac{1}{N(s)} \sum_{i=1}^{N(s)} G^{(i)}_t
\]
where $G^{(i)}_t$ is the return after the $i$-th visit to state $s$, and $N(s)$ is the number of such visits. By the law of large numbers, this average converges to the expected return as the number of episodes grows \cite{sutton2018reinforcement}.

There are different strategies to determine which returns should be included in the averaging process. A commonly used one is the \textbf{first-visit Monte Carlo} method, which updates the value of a state only using the first time it appears in each episode. This helps reduce the correlation between multiple returns from the same state within a single episode. We describe it in Algorithm \ref{algo:first_visit}.

\begin{algorithm}[ht]
\caption{First-Visit Monte Carlo Prediction}
\begin{algorithmic}[1]
\Require A policy $\pi$ to be evaluated
\State Initialize $V(s) \in \R$, arbitrarily, for all $s \in \mathcal{S}$
\State Initialize \textit{Returns}$(s) \leftarrow$ an empty list, for all $s \in \mathcal{S}$ 
\For{each episode}
    \State Generate an episode: $s_0, a_0, r_1, \ldots, s_T$
    \State $G \leftarrow 0$
    \For{$t = T-1$ \textbf{to} $0$}
        \State $G \leftarrow \gamma G + r_{t+1}$
        \If{$s_t$ does not appears in $s_0, s_1, \ldots, s_{t-1}$}
            \State Append G to \textit{Returns}$(s_t)$
            \State $V(s_t) \leftarrow$ average$($\textit{Returns}$(s_t))$
        \EndIf
    \EndFor
\EndFor
\end{algorithmic}
\label{algo:first_visit}
\end{algorithm}

Despite its simplicity, this strategy exemplifies how Monte Carlo prediction can be implemented without requiring knowledge of transition dynamics, and how returns can be used directly to evaluate policies. Alternative strategies, such as every-visit MC or weighted averaging, can be applied depending on the characteristics of the environment and the desired bias-variance trade-off~\cite{sutton2018reinforcement}.
\section{Temporal Difference Learning}
\label{sec:td_learning}

In \gls{rl}, \gls{td} learning refers to a class of methods that estimate value functions by \emph{bootstrapping}. Instead of waiting for the final outcome of an episode, as in Monte Carlo methods, TD methods update value estimates based on other learned estimates—specifically, using the observed reward and the estimated value of the next state.

At each time step, the learning update relies on the difference between the current value estimate and a one-step ``\textit{lookahead}'' estimate of the future return. This difference is known as the \textbf{TD error}, defined as:
\begin{equation}
    \label{eq:td_error}
    \delta_t = r_{t+1} + \gamma V(s_{t+1}) - V(s_t),
\end{equation}

\noindent where $t$ is the time step, $r_{t+1}$ is the reward received after taking an action at step $t$, $\gamma$ is the discount factor, and $V(s)$ denotes the estimated value function.

The simplest TD method, called \textbf{$TD(0)$} or \emph{one-step TD}, performs the following update immediately after the transition to state $S_{t+1}$ and receiving reward $R_{t+1}$:
\begin{equation}
    \label{eq:td_update}
    V(S_t) \gets V(S_t) + \alpha \left[ R_{t+1} + \gamma V(S_{t+1}) - V(S_t) \right],
\end{equation}

\noindent where $\alpha$ is the learning rate. This method is called $TD(0)$ because it bootstraps using only the next state’s value estimate.

We summarize the $TD(0)$ procedure for policy evaluation in Algorithm~\ref{alg:td_0}.

\begin{algorithm}
\caption{Tabular TD(0) for estimating $v_\pi$}
\label{alg:td_0}
\begin{algorithmic}
\Require Policy $\pi$ to be evaluated
\Require Step size $\alpha \in (0,1]$
\State Initialize $V(s)$ for all $s \in \mathcal{S}$ arbitrarily (except $V(\text{terminal}) = 0$)
\For {each episode}
    \State Initialize $S$
    \For {each time step $t$ of the episode}
        \State $A \gets \text{action sampled from } \pi(S_t)$
        \State Take action $A$, observe $R_{t+1}, S_{t+1}$
        \State $V(S_t) \gets V(S_t) + \alpha \left[ R_{t+1} + \gamma V(S_{t+1}) - V(S_t) \right]$
        \State $S_t \gets S_{t+1}$
    \EndFor
\EndFor
\end{algorithmic}
\end{algorithm}

Many value-based and policy gradient methods are built upon the TD-learning framework. They extend the TD idea to estimate action-value functions, which are then used to derive policies~\cite{sutton2018reinforcement}. 

\subsection{Value-Based Methods}
\label{sec:value_based}

Value-based methods in \gls{rl} are a class of algorithms that estimate how good it is to take a certain action in a given state. Instead of learning policies directly, these methods derive them indirectly by estimating value functions and acting greedily with respect to those estimates.

The most fundamental value function in this context is the \textbf{action-value function}, or $Q$-function, which is defined for a policy $\pi$ as:
\begin{equation}
    Q^\pi(s, a) = \mathbb{E}_\pi \left[ \sum_{k=0}^{\infty} \gamma^k r_{t+k+1} \,\bigg|\, s_t = s, a_t = a \right],
\end{equation}

\noindent representing the expected return when starting from state $s$, taking action $a$, and thereafter following policy $\pi$. The optimal action-value function $Q^*(s, a)$ maximizes this expectation over all possible policies.

\subsubsection{Q-Learning}
Q-Learning~\cite{watkins1989learning} is an \emph{off-policy} TD control algorithm that estimates $Q^*(s, a)$ directly. It updates its estimates using the Bellman optimality equation:

\begin{equation}
    \label{eq:q_learning_update}
    Q(s_t, a_t) \gets Q(s_t, a_t) + \alpha \left[ r_{t+1} + \gamma \max_{a'} Q(s_{t+1}, a') - Q(s_t, a_t) \right],
\end{equation}

\noindent where $\alpha$ is the learning rate, $\gamma$ is the discount factor, and the term inside the brackets is known as the \emph{TD target}. Q-Learning builds an estimate of $Q^*$ regardless of the policy being followed during training, making it suitable for off-policy learning. The resulting greedy policy can be extracted as:

\begin{equation}
    \pi^*(s) = \arg\max_a Q^*(s, a).
\end{equation}

To ensure sufficient exploration during learning, an $\varepsilon$-greedy strategy is typically used: the agent selects a random action with probability $\varepsilon$, and the greedy action otherwise.

Algorithm~\ref{alg:q_learning} shows the basic Q-Learning algorithm for finite state and action spaces.

\begin{algorithm}
\caption{Tabular Q-Learning}
\label{alg:q_learning}
\begin{algorithmic}
\Require Step size $\alpha \in (0,1]$, exploration rate $\varepsilon$, discount factor $\gamma$
\State Initialize $Q(s, a)$ arbitrarily for all $s \in \mathcal{S}$ and $a \in \mathcal{A}(s)$
\For {each episode}
    \State Initialize $\mathcal{S}$
    \For {each time step $t$ of the episode}
        \State With probability $\varepsilon$, select $a_t$ randomly
        \State Otherwise, $a_t \gets \arg\max_a Q(s_t, \mathcal{A})$
        \State Take action $a_t$, observe $r_{t+1}, s_{t+1}$
        \State $Q(s_t, a_t) \gets Q(s_t, a) + \alpha \left[ r_{t+1} + \gamma \max_{a'} Q(s_{t+1}, a_{t+1}) - Q(s_t, a_t) \right]$
        \State $s_t \gets s_{t+1}$
    \EndFor
\EndFor
\end{algorithmic}
\end{algorithm}

Although Q-Learning is guaranteed to converge under certain conditions in tabular settings, it becomes infeasible for large or continuous state spaces due to the curse of dimensionality. This limitation motivated the development of function approximation methods, most notably the \gls{dqn}~\cite{mnih2015human}.

\subsubsection{Deep Q-Networks}
\label{sec:dqn}

As traditional Q-Learning relies on tabular representations, it becomes impractical in environments with large or continuous state spaces. To address this limitation, \cite{mnih2015human} introduced the \gls{dqn}, a framework that combines Q-learning with deep neural networks to approximate the action-value function.

The \gls{dqn} builds upon earlier work such as TD-Gammon~\cite{tesauro1994td}, where a multilayer perceptron (MLP) with a single hidden layer was used to approximate the value function in the TD($\lambda$) framework~\cite{sutton2018reinforcement}. However, DQN represents a major breakthrough by successfully integrating deep learning into reinforcement learning through three key innovations:

\begin{itemize}
    \item \textbf{Convolutional State Representation:} When the agent receives high-dimensional sensory inputs like images, DQN uses Convolutional Neural Networks (CNNs)~\cite{goodfellow2013multi} to extract compact, informative state representations directly from raw pixels.
    
    \item \textbf{Experience Replay:} DQN introduces a memory buffer, often referred to as a \gls{rb}, that stores past transitions $(s, a, r, s')$. During training, mini-batches of these experiences are sampled uniformly at random to update the network. This breaks temporal correlations between samples and improves sample efficiency.
    
    \item \textbf{Target Networks:} To address instability and divergence issues during training, DQN employs a separate target network. This network is a delayed copy of the current policy network and is used to compute the TD target during Q-learning updates. By holding the target values fixed for a number of steps, DQN mitigates the problem of moving targets and reduces the overestimation bias in Q-values~\cite{dqnTarget}.
\end{itemize}

The update rule for DQN is a modified version of the Q-learning update:

\begin{equation}
    \label{eq:dqn_update}
    Q(s_t, a_t; \theta) \leftarrow Q(s_t, a_t; \theta) + \alpha \big[ r_{t+1} + \gamma \max_{a'} Q(s_{t+1}, a'; \theta^-) - Q(s_t, a_t; \theta) \big],
\end{equation}

\noindent where $\theta$ represents the parameters of the policy network and $\theta^-$ those of the target network. The parameters $\theta^-$ are periodically updated to match $\theta$ according to a fixed synchronization interval.

These innovations allowed DQN to learn policies directly from pixels and achieve human-level performance on various Atari 2600 games~\cite{mnih2015human}. Figure~\ref{fig:dqn_architecture} illustrates the core architecture of the DQN agent, while Algorithm \ref{alg:dqn} describes it's training process.

\begin{figure}[ht]
    \caption{Schematic of Deep Q-Network (DQN): images are processed through a CNN to extract features, which are then mapped to Q-values for each action. The target network stabilizes training by providing fixed TD targets.}
    \centering
    \includegraphics[width=0.75\textwidth]{images/fundamentacao/rl/dqn.jpeg}
    \label{fig:dqn_architecture}
    \par\medskip\ABNTEXfontereduzida\selectfont\textbf{Source:} Author
\end{figure}

\begin{algorithm}
\caption{Deep Q-Network}
\label{alg:dqn}
\begin{algorithmic}[1]
\Require Initialize Q-network with weights $\theta$
\Require Initialize target network with weights $\theta^- \leftarrow \theta$
\Require Initialize replay buffer $\mathcal{D}$
\For{each episode}
    \State Initialize $s$
    \For{each step $t$ in the episode}
        \State Select action $a_t$ using an $\epsilon$-greedy policy from $Q(s, a; \theta)$
        \State Execute $a_t$ and observe reward $r_{t+1}$ and next state $s_{t+1}$
        \State Store transition $(s_t, a_t, r_{t+1}, s_{t+1})$ in $\mathcal{D}$
        \State Sample random mini-batch of transitions $(s_j, a_j, r_j, s_j')$ from $\mathcal{D}$
        \State Compute TD target:
        \[
        y_j = r_j + \gamma \max_{a'} Q(s_j', a'; \theta^-)
        \]
        \State Perform gradient descent on:
        \[
        \left(y_j - Q(s_j, a_j; \theta)\right)^2
        \]
        \State Every $C$ steps, update target network: $\theta^- \leftarrow \theta$
        \State $s \leftarrow s_{t+1}$
    \EndFor
\EndFor
\end{algorithmic}
\end{algorithm}

\subsection{Policy Gradient Methods}
\label{sec:pg}
\gls{pg} methods employ a parameterized policy to select actions directly, without relying on an action-value function during execution. We denote the policy parameters as $\theta \in \R^n$. These parameters are updated throughout training to maximize the expected cumulative reward. Although value functions can be used to guide the update of $\theta$, they are typically not consulted for action selection. The probability of choosing action $a$ in state $s$ at time $t$ under the parameter vector $\theta$ is denoted as:
\[
\pi(a|s, \theta) \doteq Pr\{A_t = a \mid S_t = s, \theta_t = \theta\}.
\]

\gls{pg} methods optimize the policy by estimating the gradient of a performance measure $J(\theta)$ with respect to the policy parameters. The objective is to perform gradient ascent in $J$, following the update rule:

\begin{equation}
\label{eq:gradient_ascent}
    \theta_{t+1} \doteq \theta + \lambda\widehat{\nabla J(\theta_t)},
\end{equation}

\noindent
where $\widehat{\nabla J(\theta_t)}$ is a stochastic estimate of the true gradient. Any method that follows this general framework is considered a policy gradient method. The core theoretical result guiding these updates is summarized in Theorem~\ref{theo:pg_theorem}.

\begin{theorem}[Policy Gradient Theorem]
\label{theo:pg_theorem}
    \begin{gather}
    \begin{split}
    \nabla_\theta J (\pi_\theta) &= \int_S p^\pi(s) \int_A \nabla_\theta \pi_\theta (a|s) Q_\pi (s,a) \, da \, ds \\
    &= \E_{S\sim p^\pi,a\sim \pi_\theta}[\nabla_\theta \log\pi_\theta (a|s) Q_\pi (s,a)]
    \end{split}
    \label{eq:pg_theorem}
    \end{gather} 
\end{theorem}



\subsubsection{Actor-Critic}
\label{sec:actor_critic}
In this work, we employ an \textbf{Actor-Critic} algorithm, which combines the strengths of \gls{pg} methods and value-based approaches. The \textbf{actor} represents the policy $\pi_\theta(a|s)$, updated using the gradient from Theorem~\ref{theo:pg_theorem}, while the \textbf{critic} estimates the action-value function $Q_\pi(s,a)$ through a parameterized function $Q_w(s,a)$. The critic provides feedback to the actor by evaluating its actions.

Figure~\ref{fig:actor_critic} illustrates the general training loop of an Actor-Critic architecture.

\begin{figure}[ht]
    \caption{Example of Actor-Critic training loop. The $Q$-values of the critic module are used to train the actor module.}
    \centering
    \includegraphics[width=0.75\textwidth]{images/fundamentacao/rl/actor_critic.png}
    \par\medskip\ABNTEXfontereduzida\selectfont\textbf{Source:} Author
    \label{fig:actor_critic}
\end{figure} 

The key distinction between Actor-Critic and standard \gls{pg} methods lies in the use of bootstrapping. Rather than relying on the true value $Q_\pi(s,a)$—which is often unknown—the critic substitutes it with a learned approximation $Q_w(s,a)$. This approximation introduces bias and its effectiveness depends on the quality of the function approximator.

However, under specific conditions, this bias can be eliminated. If the approximator satisfies $Q_w(s,a) = \nabla_\theta \log \pi_\theta(a|s)^T w$ and the parameters $w$ minimize the mean squared error:
\[
\epsilon^2(w) = \E_{s \sim p, a \sim \pi_\theta}\left[(Q_w(s,a) - Q_\pi(s,a))^2\right],
\]
then the update remains unbiased \cite{sutton2018reinforcement}.
\subsubsection{Deterministic Policy Gradient Methods}
\label{sec:dpg}

The \gls{dpg} algorithm was introduced as an \textbf{off-policy} Actor-Critic method capable of learning a deterministic target policy from data generated by a separate behavior policy \cite{dpg}. This approach extends the ideas from \cite{offpac}, in which a behavior policy $\beta(a|s)$, different from the target policy $\pi_{\theta}(a|s)$, is used to generate trajectories for policy updates.

\gls{dpg} maintains a parameterized deterministic policy, denoted $\mu(s|\theta_\mu)$, which maps each state to a specific action. The critic, $Q_w(s,a)$, estimates the action-value function and is updated using the Bellman equation (Equation~\ref{eq:bellman_value}). The actor is trained to maximize the expected return by applying the deterministic policy gradient:

\begin{gather}
\begin{split}
\label{eq:dpg}
\nabla_{\theta_\mu}J &\approx \E_{s_t \sim p^\beta}\left[\nabla_{\theta_\mu}Q(s,a|\theta_Q)\big|_{s=s_t, a=\mu(s_t)}\right] \\
&= \E_{s_t \sim p^\beta}\left[\nabla_{a}Q(s,a|\theta_Q)\big|_{s=s_t, a=\mu(s_t)} \cdot \nabla_{\theta_\mu}\mu(s|\theta_\mu)\big|_{s=s_t}\right],
\end{split}
\end{gather}

\noindent
where $p^\beta$ denotes the state distribution under the behavior policy $\beta$.

While \gls{dpg} is effective in continuous action spaces, its performance is often limited by the function approximator’s capacity to estimate $Q$ accurately—especially in high-dimensional state spaces. To address this, \cite{ddpg} proposed the \gls{ddpg} algorithm, inspired by the \gls{dqn} architecture. \gls{ddpg} incorporates deep neural networks to approximate both the actor and the critic, and leverages Experience Replay for sample efficiency and stability.

In addition to these features, \gls{ddpg} introduces several techniques to stabilize training:

- \textbf{Target Networks:} As in \gls{dqn} \cite{dqnTarget}, separate target networks are used for both the actor and critic to prevent harmful feedback loops. These target networks are updated slowly using a soft update rule: 
  \[
  \theta' \gets \tau\theta + (1 - \tau)\theta', \quad 0 < \tau \ll 1,
  \]
  where $\theta'$ are the target weights, and $\theta$ are the main network’s weights.

- \textbf{Exploration Noise:} Since \gls{ddpg} learns a deterministic policy, it requires an external mechanism for exploration. The algorithm employs an Ornstein-Uhlenbeck process \cite{uhlenbeck1930theory} to generate temporally correlated noise, which is added to the actions during training. 
\subsection{Soft Actor-Critic}
\label{sec:sac}

The \gls{sac} algorithm extends the deterministic actor-critic approach by addressing some of the limitations of \gls{ddpg}, such as poor exploration and training instability \cite{sac}. While \gls{ddpg} uses a deterministic policy and requires carefully tuned exploration noise, \gls{sac} employs a \emph{stochastic policy} and introduces the concept of \emph{entropy maximization} as part of the optimization objective.

The key idea behind \gls{sac} is to encourage exploration by maximizing a trade-off between expected return and the entropy of the policy. Formally, the objective becomes:

\begin{equation}
\label{eq:sac_objective}
J(\pi) = \sum_{t=0}^{T} \E_{(s_t,a_t)\sim \rho_\pi} \left[ r(s_t,a_t) + \alpha \mathcal{H}(\pi(\cdot|s_t)) \right],
\end{equation}

\noindent
where $\mathcal{H}(\pi(\cdot|s_t)) = \E_{a_t \sim \pi}[-\log \pi(a_t|s_t)]$ is the entropy of the policy at state $s_t$, and $\alpha$ is a temperature parameter that controls the relative importance of the entropy term.

By encouraging higher entropy (i.e., more randomness), SAC ensures better exploration and robustness during training. Unlike \gls{ddpg}, \gls{sac} can learn from a wider range of trajectories without collapsing to suboptimal deterministic behaviors.

\gls{sac} introduces several innovations over \gls{ddpg}:

\begin{itemize}
    \item \textbf{Stochastic Policies:} Instead of outputting a single deterministic action, SAC learns a policy $\pi(a|s)$ that outputs a distribution over actions, typically modeled by a squashed Gaussian.
    
    \item \textbf{Entropy-Regularized Objective:} As seen in Equation~\ref{eq:sac_objective}, SAC maximizes both expected reward and entropy, making the agent more exploratory and less likely to get stuck in poor local optima.
    
    \item \textbf{Twin Q-networks:} \gls{sac} uses two critic networks to mitigate overestimation bias. The minimum of the two Q-values is used in the target computation.
    
    \item \textbf{Automatic Temperature Adjustment:} \gls{sac} can learn the temperature parameter $\alpha$ automatically by adjusting it to match a target entropy level. This removes the need for manual tuning of exploration-exploitation trade-offs.
\end{itemize}

These enhancements make \gls{sac} highly sample-efficient and robust in continuous control tasks. It has shown strong empirical performance across a variety of benchmark environments, often outperforming \gls{ddpg} in both sample efficiency and final performance.

\begin{algorithm}
    \caption{Soft Actor-Critic (SAC) Algorithm}
    \begin{algorithmic}
        \State Initialize actor parameters $\theta$, critic parameters $w_1$, $w_2$, temperature parameter $\alpha$
        \State Initialize target critic parameters $w_1' \gets w_1$, $w_2' \gets w_2$
        \State Initialize empty replay buffer $\mathcal{D}$
        \For{episode = 1 to M}
            \State Receive initial state $s_0$
            \For{$t = 1$ to T}
                \State Sample action $a_t \sim \pi_\theta(\cdot|s_t)$
                \State Execute action $a_t$, observe reward $r_t$ and next state $s_{t+1}$
                \State Store transition $(s_t, a_t, r_t, s_{t+1})$ in $\mathcal{D}$
                
                \State Sample minibatch $\{(s_i, a_i, r_i, s_{i+1})\}_{i=1}^N$ from $\mathcal{D}$
                \State Sample $a_{i+1} \sim \pi_\theta(\cdot|s_{i+1})$ and compute $\log \pi_\theta(a_{i+1}|s_{i+1})$
                \State $y_i \gets r_i + \gamma \left( \min_{j=1,2} Q_{w_j'}(s_{i+1}, a_{i+1}) - \alpha \log \pi_\theta(a_{i+1}|s_{i+1}) \right)$
                
                \State Update critics by minimizing:
                \[
                L(w_j) = \frac{1}{N} \sum_i (Q_{w_j}(s_i, a_i) - y_i)^2, \quad \text{for } j = 1, 2
                \]
                
                \State Update actor by minimizing:
                \[
                J(\theta) = \frac{1}{N} \sum_i \left( \alpha \log \pi_\theta(a_i|s_i) - \min_{j=1,2} Q_{w_j}(s_i, a_i) \right)
                \]

                \State Update temperature (optional, if $\alpha$ is learned):
                \[
                J(\alpha) = \frac{1}{N} \sum_i \alpha \left( -\log \pi_\theta(a_i|s_i) - \mathcal{H}_{\text{target}} \right)
                \]

                \State Soft update target networks:
                \[
                w_j' \gets \tau w_j + (1 - \tau) w_j', \quad \text{for } j = 1, 2
                \]
            \EndFor
        \EndFor
    \end{algorithmic}
    \label{algo:sac}
\end{algorithm}

The methods reviewed in this chapter differ in their learning dynamics but share a fundamental reliance on the reward signal as the central driver of optimization. In this sense, reward design and prioritization constitute a common challenge across all frameworks. DyLam is designed as an algorithm-agnostic mechanism and can be combined with each of these methods without modifying their underlying update rules. By decoupling reward prioritization from the learning algorithm, DyLam provides an unified and principled way to handle decomposed reward structures across different Reinforcement Learning paradigms.
\subsection{Farama Gymnasium}
\label{sec:gym}

The \textbf{Farama Gymnasium} is a standardized Python library for developing and interacting with reinforcement learning environments~\cite{towers2024gymnasium}. It is the spiritual successor to OpenAI's Gym, which has been a cornerstone for the RL community since its release. With the deprecation of the original Gym repository, the Farama Foundation took the responsibility of maintaining and extending its functionality under the name Gymnasium\footnote{\url{https://github.com/Farama-Foundation/Gymnasium}}. This pattern has become ubiquitous in RL workflows, enabling efficient testing, benchmarking, and integration across tools~\cite{mnih2015human, dqnTarget, ddpg, sac}.

The core idea behind Gymnasium is to provide a unified API for RL environments, enabling researchers and developers to experiment with a wide range of tasks through a common interface. This standardization facilitates reproducibility, benchmarking, and interoperability between agents, environments, and libraries.

\subsubsection{The Environment Interface Pattern}
Gymnasium enforces a clear abstraction separating the \emph{environment} from the \emph{simulator}. The environment handles the agent-environment loop, while the simulator (e.g., physics engines or emulators) encapsulates the dynamics and rendering. This abstraction allows a consistent API regardless of the domain—be it gridworlds~\cite{MinigridMiniworld23}, robotics~\cite{todorov2012mujoco}, or Atari games~\cite{bellemare13arcade}.

To define a new environment in Gymnasium, several components must be specified:

\begin{itemize}
    \item \textbf{Observation space:} Defines the structure and type of states returned to the agent. This could be discrete labels, continuous vectors, images, etc.
    \item \textbf{Action space:} Specifies the valid actions an agent can take. Gymnasium supports discrete, continuous, and compound spaces.
    \item \textbf{Reward signal:} A scalar or vector indicating the immediate feedback for the agent’s action. The semantics are domain-specific but must follow the Gymnasium interface.
    \item \textbf{Core methods:}
    \begin{itemize}
        \item \texttt{reset()}: Initializes the environment to a starting state, returning the initial observation.
        \item \texttt{step(action)}: Advances the environment one timestep given the action. It returns a tuple of $(\text{observation}, \text{reward}, \text{terminated}, \text{truncated}, \text{info})$.
    \end{itemize}
\end{itemize}

\subsubsection{Registration and Wrappers}
Gymnasium uses a registry system to manage available environments, allowing users to instantiate them using a string identifier. Additionally, it provides \emph{wrappers}, which are modular tools for modifying environments (e.g., observation normalization, reward shaping, or frame skipping) without changing their core logic.

\subsubsection{Basic Usage Example}
The following Python snippet demonstrates the basic usage of a Gymnasium environment using the \texttt{LunarLander-v3} domain. It illustrates environment creation, stepping through the simulation, and resetting after episode termination:

\begin{lstlisting}[language=Python, caption=Basic usage of Gymnasium, label=code:gym]
import gymnasium as gym

# Initialise the environment
env = gym.make("LunarLander-v3", render_mode="human")

# Reset the environment to generate the first observation
observation, info = env.reset(seed=42)
for _ in range(1000):
    # This is where you would insert your policy
    action = env.action_space.sample()

    # Step through the environment with the action
    observation, reward, terminated, truncated, info = env.step(action)

    # If the episode has ended, reset to start a new episode
    if terminated or truncated:
        observation, info = env.reset()

env.close()
\end{lstlisting}


