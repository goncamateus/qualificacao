\section{Reinforcement Learning}
\label{sec:rl}

Reinforcement Learning addresses the problem of learning how an autonomous decision-making entity, referred to as an \emph{agent}, should behave when interacting with an \emph{environment} in order to achieve a given objective. This interaction unfolds over discrete time steps. At each time step $t$, the agent observes the current \emph{state} of the environment, denoted by $s_t \in \mathcal{S}$, selects an \emph{action} $a_t \in \mathcal{A}$, and, as a consequence, the environment transitions to a new state $s_{t+1}$ while emitting a scalar \emph{reward} $r_t$ that evaluates the immediate outcome of the agent’s decision~\cite{sutton2018reinforcement}.

The reward signal encodes the task objective and constitutes the only form of direct feedback available to the agent. Importantly, actions influence not only the immediate reward but also future states and subsequent rewards, making reinforcement learning inherently a sequential decision-making problem under uncertainty. The agent must therefore reason about long-term consequences rather than relying solely on instantaneous feedback.

The agent’s behavior is formalized through a \emph{policy}, denoted by $\pi$, which specifies how actions are selected in each state. In stochastic settings, a policy is represented as a conditional probability distribution $\pi(a \mid s)$, whereas in deterministic settings it corresponds to a direct mapping $\mu(s) = a$. The central goal of reinforcement learning is to learn a policy that maximizes the expected cumulative reward over time, commonly referred to as the \emph{return} and denoted by $G$. Thus, RL can be understood as the process of learning a behavior strategy that prescribes how the agent should act in every possible situation it may encounter.

\begin{figure}[ht]
\centering
\caption{Agent-environment interaction loop in Reinforcement Learning.}
\includegraphics[width=0.6\linewidth]{images/fundamentacao/rl/rl-loop.png}
\par\medskip\ABNTEXfontereduzida\selectfont\textbf{Source:} Author
\label{fig:rl_loop}
\end{figure}

To analyze the agent–environment interaction in a principled manner, reinforcement learning problems are commonly embedded within a formal sequential decision-making framework. This framework specifies how the environment responds to the agent’s actions, how rewards are generated, and how future outcomes depend on current decisions. It also establishes the assumptions under which learning and optimization are well defined, such as the dependence of future states on the current state and action rather than on the full interaction history. The precise mathematical structure underlying this framework—together with its associated assumptions and properties—is presented in a dedicated section through the Markov Decision Process formalism. At this stage, it is sufficient to recognize that this abstraction underpins the majority of theoretical results and algorithmic developments in reinforcement learning, providing a common foundation for the methods discussed in the remainder of this chapter.

\section{Finite Markov Decision Process}
\label{sec:mdp}

The \gls{mdp} is a mathematical framework used to formalize sequential decision-making problems in \gls{rl}, where the agent learns from incremental feedback. It provides a well-defined structure to model the interaction between an agent and its environment. As previously introduced, the agent observes the environment, takes actions, and receives feedback in the form of rewards. This interactive process, at its core, defines a \gls{mdp}~\cite{sutton2018reinforcement}.

\begin{figure}
    \caption{Representation of a \gls{mdp}. Source: Author}
    \centering
    \includegraphics[width=0.75\linewidth]{images/fundamentacao/rl/mdp.png}
    \label{fig:fundamentacao/mdp_basic}
    \par\medskip\ABNTEXfontereduzida\selectfont\textbf{Source:} Author
\end{figure}

The objective of the agent in an MDP is to maximize the expected \emph{return}, which quantifies the cumulative reward obtained over time. Formally, the return from time step $t$ is defined as
\begin{equation}
    \label{eq:return}
    G_t = r_{t+1} + \gamma r_{t+2} + \gamma^2 r_{t+3} + \cdots = \sum_{k=0}^\infty \gamma^k r_{t+k+1}
\end{equation}
where $r_{t+k+1}$ denotes the reward received at future time step $t+k+1$, and $\gamma \in [0,1]$ is the \emph{discount factor}. The discount factor controls the relative importance of future rewards with respect to immediate ones. Values of $\gamma$ close to zero bias the agent toward short-term rewards, whereas values close to one encourage long-term planning by assigning higher weight to delayed outcomes.

With this notion of return, a finite Markov Decision Process is formally defined by the tuple $(\mathcal{S}, \mathcal{A}, \mathcal{P}, \mathcal{R}, \gamma)$, where $\mathcal{S}$ is a finite set of states, $\mathcal{A}$ is a finite set of actions, $\mathcal{P}(s' \mid s,a)$ denotes the state transition probability from state $s$ to state $s'$ after taking action $a$, and $\mathcal{R}(s,a)$ represents the expected immediate reward associated with the state–action pair $(s,a)$.

At each discrete time step $t = 0, 1, 2, \dots$, the agent observes the current state $s_t \in \mathcal{S}$, selects an action $a_t \in \mathcal{A}(s_t)$, receives a scalar reward $r_{t+1} \in \mathbb{R}$, and transitions to a new state $s_{t+1} \in \mathcal{S}$ according to the transition probability $\mathcal{P}(s_{t+1} \mid s_t, a_t)$. This interaction loop, illustrated in Figure~\ref{fig:fundamentacao/mdp_basic}, continues until a terminal condition is reached or indefinitely in continuing tasks.




\section{Bellman Equations}
\label{sec:bellman}

The formal definition of a \gls{mdp} allows us to reason about the quality of states and actions through \emph{value functions}. These functions quantify the expected long-term return obtained by an agent when interacting with the environment under a given policy. The recursive relationships that characterize such functions are known as the \emph{Bellman equations}~\cite{sutton2018reinforcement}.

Originally introduced by Richard Bellman in the context of dynamic programming~\cite{bellman1966dynamic}, these equations express a fundamental principle: the value of a decision can be decomposed into the immediate reward obtained and the value of subsequent decisions. This recursive structure enables reasoning about long-term consequences without explicitly enumerating all future trajectories.

Let $\pi: \mathcal{S} \rightarrow \mathcal{A}$ denote a (possibly stochastic) policy. The \textbf{state-value function} $v_\pi(s)$ is defined as the expected return when the agent starts in state $s$ and follows policy $\pi$ thereafter:
\begin{equation}
v_\pi(s) = \mathbb{E}_\pi \left[ \sum_{t=0}^{\infty} \gamma^t r_{t+1} \,\middle|\, s_0 = s \right].
\end{equation}

By conditioning on the first action selected under $\pi$ and the subsequent state transition, this definition induces a recursive relationship between the value of a state and the values of its successor states. This relationship is formalized by the \textbf{Bellman expectation equation}, which the state-value function satisfies:
\begin{equation}
v_\pi(s) = \sum_{a \in \mathcal{A}} \pi(a|s)
\sum_{s' \in \mathcal{S}} \mathcal{P}(s'|s,a)
\left[ \mathcal{R}(s,a) + \gamma v_\pi(s') \right].
\label{eq:bellman_value}
\end{equation}

Analogously, the \textbf{action-value function} $q_\pi(s,a)$ represents the expected return obtained by taking action $a$ in state $s$ and then following policy $\pi$:
\begin{equation}
q_\pi(s,a) = \mathbb{E}_\pi \left[ \sum_{t=0}^{\infty} \gamma^t r_{t+1}
\,\middle|\, s_0 = s, a_0 = a \right].
\end{equation}

Applying the same recursive reasoning yields the Bellman expectation equation for the action-value function:
\begin{equation}
q_\pi(s,a) =
\sum_{s' \in \mathcal{S}} \mathcal{P}(s'|s,a)
\left[
\mathcal{R}(s,a)
+ \gamma \sum_{a' \in \mathcal{A}} \pi(a'|s') q_\pi(s',a')
\right].
\label{eq:bellman_action_value}
\end{equation}

While these equations characterize the expected return under a fixed policy, the central objective in reinforcement learning is to identify an \emph{optimal policy} $\pi^*$ that maximizes the expected return from every state. This leads to the definition of the \textbf{optimal value functions}:
\begin{align}
v_*(s) &= \max_{\pi} v_\pi(s), \\
q_*(s,a) &= \max_{\pi} q_\pi(s,a).
\end{align}

These functions satisfy the \textbf{Bellman optimality equations}, in which the expectation over actions is replaced by a maximization operator. The optimal state-value function obeys:
\begin{equation}
v_*(s) =
\max_{a \in \mathcal{A}}
\sum_{s' \in \mathcal{S}} \mathcal{P}(s'|s,a)
\left[ \mathcal{R}(s,a) + \gamma v_*(s') \right],
\label{eq:bellman_opt_value}
\end{equation}
and the optimal action-value function satisfies:
\begin{equation}
q_*(s,a) =
\sum_{s' \in \mathcal{S}} \mathcal{P}(s'|s,a)
\left[ \mathcal{R}(s,a) + \gamma \max_{a' \in \mathcal{A}} q_*(s',a') \right].
\label{eq:bellman_opt_q}
\end{equation}

An optimal policy can then be recovered by acting greedily with respect to $q_*$:
\begin{equation}
\pi^*(s) = \arg\max_{a \in \mathcal{A}} q_*(s,a).
\end{equation}

Although the Bellman equations provide an exact characterization of optimal behavior, solving them directly requires complete knowledge of the environment dynamics and is computationally intractable for large or continuous state and action spaces~\cite{mnih2015human, ddpg, tesauro1994td}. These limitations motivate the development of approximate, sample-based, and function-approximation methods, which form the basis of the reinforcement learning algorithms discussed in the subsequent sections.

\section{Monte Carlo Learning}
\label{sec:monte_carlo}

\gls{mc} methods are a foundational class of model-free reinforcement learning techniques that estimate value functions and optimize policies by relying exclusively on sampled experience. The term ``Monte Carlo'' refers to methods that solve problems through random sampling — a concept originating from simulations used in physics and mathematics, and named after the famous casino in Monaco due to their reliance on stochasticity and chance \cite{kroese2014monte}. In reinforcement learning, this stochastic element comes from the interaction between the agent and the environment, where the agent collects data through episodes and averages the observed returns.

Unlike dynamic programming, which requires full knowledge of the environment’s transition and reward functions, Monte Carlo methods operate without access to a model. Instead, they use complete sequences of experience — also known as episodes — to estimate the expected return from states or state-action pairs. These experiences are formed by tuples of the form $(s, a, r, s')$, representing a single interaction between the agent and the environment.

The experience collected can be either \textit{real}, obtained through direct interaction with the environment, or \textit{simulated}, produced by a learned or hand-crafted model. Regardless of the source, Monte Carlo methods assume that these episodes are representative of the agent’s behavior under a fixed policy and that they can be used to estimate long-term returns by averaging the observed outcomes.

Because Monte Carlo methods require the return to be computed over an entire episode, they are naturally suited for \textbf{episodic tasks}. In such cases, policy updates typically occur at the end of each episode. For continuing tasks, updates are often performed every fixed number of steps, simulating artificial episodes to approximate returns. This episodic nature introduces a key characteristic of \gls{mc} methods: they are incremental only in an \textit{episode-by-episode} sense, but not in a \textit{step-by-step} fashion~\cite[p.~91]{sutton2018reinforcement}.

However, this reliance on episodic returns also introduces challenges. One of them is the \textbf{nonstationarity} of the return distribution: as the policy changes over time, previously collected experiences may no longer reflect the behavior of the current policy, making naive averaging potentially biased. For this reason, care must be taken when applying \gls{mc} methods in non-stationary or off-policy settings, often requiring additional tools such as importance sampling to correct for distributional shifts \cite{sutton2018reinforcement}.

Once the agent collects complete episodes under a fixed policy $\pi$, the next step is to estimate the value of states encountered during these episodes. This process is known as \textbf{Monte Carlo prediction}, and its goal is to compute the state-value function $v_\pi(s)$ using empirical returns.

The return, denoted by $G_t$, represents the cumulative reward obtained by the agent from time step $t$ onward. For a given trajectory $(S_0, A_0, R_1, \ldots, S_T)$, the return is defined as:
\begin{equation}
    G_t = R_{t+1} + \gamma R_{t+2} + \gamma^2 R_{t+3} + \cdots = \sum_{k=0}^\infty \gamma^k R_{t+k+1}
\end{equation}
In episodic tasks, where the episode eventually terminates, this summation is finite and can be computed exactly.

To estimate $v_\pi(s)$, Monte Carlo methods average the observed returns following visits to state $s$ across multiple episodes. More formally, we use:
\[
v_\pi(s) \approx \frac{1}{N(s)} \sum_{i=1}^{N(s)} G^{(i)}_t
\]
where $G^{(i)}_t$ is the return after the $i$-th visit to state $s$, and $N(s)$ is the number of such visits. By the law of large numbers, this average converges to the expected return as the number of episodes grows \cite{sutton2018reinforcement}.

There are different strategies to determine which returns should be included in the averaging process. A commonly used one is the \textbf{first-visit Monte Carlo} method, which updates the value of a state only using the first time it appears in each episode. This helps reduce the correlation between multiple returns from the same state within a single episode. We describe it in Algorithm \ref{algo:first_visit}.

\begin{algorithm}[ht]
\caption{First-Visit Monte Carlo Prediction}
\begin{algorithmic}[1]
\Require A policy $\pi$ to be evaluated
\State Initialize $V(s) \in \R$, arbitrarily, for all $s \in \mathcal{S}$
\State Initialize \textit{Returns}$(s) \leftarrow$ an empty list, for all $s \in \mathcal{S}$ 
\For{each episode}
    \State Generate an episode: $s_0, a_0, r_1, \ldots, s_T$
    \State $G \leftarrow 0$
    \For{$t = T-1$ \textbf{to} $0$}
        \State $G \leftarrow \gamma G + r_{t+1}$
        \If{$s_t$ does not appears in $s_0, s_1, \ldots, s_{t-1}$}
            \State Append G to \textit{Returns}$(s_t)$
            \State $V(s_t) \leftarrow$ average$($\textit{Returns}$(s_t))$
        \EndIf
    \EndFor
\EndFor
\end{algorithmic}
\label{algo:first_visit}
\end{algorithm}

Despite its simplicity, this strategy exemplifies how Monte Carlo prediction can be implemented without requiring knowledge of transition dynamics, and how returns can be used directly to evaluate policies. Alternative strategies, such as every-visit MC or weighted averaging, can be applied depending on the characteristics of the environment and the desired bias-variance trade-off~\cite{sutton2018reinforcement}.
\section{Temporal Difference Learning}
\label{sec:td_learning}

In \gls{rl}, \gls{td} learning refers to a class of methods that estimate value functions by \emph{bootstrapping}. Instead of waiting for the final outcome of an episode, as in Monte Carlo methods, TD methods update value estimates based on other learned estimates—specifically, using the observed reward and the estimated value of the next state.

At each time step, the learning update relies on the difference between the current value estimate and a one-step ``\textit{lookahead}'' estimate of the future return. This difference is known as the \textbf{TD error}, defined as:
\begin{equation}
    \label{eq:td_error}
    \delta_t = r_{t+1} + \gamma V(s_{t+1}) - V(s_t),
\end{equation}

\noindent where $t$ is the time step, $r_{t+1}$ is the reward received after taking an action at step $t$, $\gamma$ is the discount factor, and $V(s)$ denotes the estimated value function.

The simplest TD method, called \textbf{$TD(0)$} or \emph{one-step TD}, performs the following update immediately after the transition to state $S_{t+1}$ and receiving reward $R_{t+1}$:
\begin{equation}
    \label{eq:td_update}
    V(S_t) \gets V(S_t) + \alpha \left[ R_{t+1} + \gamma V(S_{t+1}) - V(S_t) \right],
\end{equation}

\noindent where $\alpha$ is the learning rate. This method is called $TD(0)$ because it bootstraps using only the next state’s value estimate.

We summarize the $TD(0)$ procedure for policy evaluation in Algorithm~\ref{alg:td_0}.

\begin{algorithm}
\caption{Tabular TD(0) for estimating $v_\pi$}
\label{alg:td_0}
\begin{algorithmic}
\Require Policy $\pi$ to be evaluated
\Require Step size $\alpha \in (0,1]$
\State Initialize $V(s)$ for all $s \in \mathcal{S}$ arbitrarily (except $V(\text{terminal}) = 0$)
\For {each episode}
    \State Initialize $S$
    \For {each time step $t$ of the episode}
        \State $A \gets \text{action sampled from } \pi(S_t)$
        \State Take action $A$, observe $R_{t+1}, S_{t+1}$
        \State $V(S_t) \gets V(S_t) + \alpha \left[ R_{t+1} + \gamma V(S_{t+1}) - V(S_t) \right]$
        \State $S_t \gets S_{t+1}$
    \EndFor
\EndFor
\end{algorithmic}
\end{algorithm}

Many value-based and policy gradient methods are built upon the TD-learning framework. They extend the TD idea to estimate action-value functions, which are then used to derive policies~\cite{sutton2018reinforcement}. 

\section{Value-Based Methods}
\label{sec:value_based}

Value-based methods in \gls{rl} are a class of algorithms that estimate the long-term utility of state-action pairs through value functions. Rather than learning a policy directly, these methods infer a policy indirectly by acting greedily with respect to the learned value estimates.

In general, value functions can be used either for \emph{policy evaluation}, where the goal is to assess the quality of a fixed policy, or for \emph{control}, where the objective is to learn an optimal policy. This distinction becomes explicit in the algorithms discussed next.

The central quantity in value-based control is the \textbf{action-value function}, or $Q$-function, defined for a policy $\pi$ as:
\begin{equation}
    Q^\pi(s, a) =
    \mathbb{E}_\pi \left[
    \sum_{k=0}^{\infty} \gamma^k r_{t+k+1}
    \,\bigg|\, s_t = s, a_t = a
    \right],
\end{equation}
which represents the expected return obtained by taking action $a$ in state $s$ and following policy $\pi$ thereafter. The optimal action-value function $Q^*(s,a)$ is defined as the maximum of this expectation over all admissible policies and induces an optimal policy by greedy action selection.

\subsection{Q-Learning}

While the formulation above applies to arbitrary policies, \emph{Q-Learning}~\cite{watkins1989learning} explicitly addresses the \emph{control} problem by directly estimating the optimal action-value function $Q^*(s,a)$. It is an \emph{off-policy} TD algorithm, meaning that it learns the value of the optimal policy independently of the policy used to generate behavior.

At each time step, Q-Learning updates its estimate according to the Bellman optimality equation:
\begin{equation}
    \label{eq:q_learning_update}
    Q(s_t, a_t) \gets Q(s_t, a_t)
    + \alpha \left[
        r_{t+1}
        + \gamma \max_{a'} Q(s_{t+1}, a')
        - Q(s_t, a_t)
    \right],
\end{equation}
where $\alpha \in (0,1]$ is the learning rate and $\gamma \in [0,1)$ is the discount factor. The expression inside the brackets corresponds to the TD error.

Because Q-Learning is off-policy, the behavior policy can be chosen to encourage exploration. In practice, an $\varepsilon$-greedy strategy is commonly adopted, in which a random action is selected with probability $\varepsilon \in [0,1]$, and the greedy action otherwise. Once learning converges, the optimal policy is obtained as:
\begin{equation}
    \pi^*(s) = \arg\max_a Q^*(s,a).
\end{equation}

Algorithm~\ref{alg:q_learning} summarizes the tabular Q-Learning procedure for finite state and action spaces.

\begin{algorithm}
\caption{Tabular Q-Learning}
\label{alg:q_learning}
\begin{algorithmic}
\Require Learning rate $\alpha \in (0,1]$, exploration rate $\varepsilon \in [0,1]$, discount factor $\gamma \in [0,1)$
\State Initialize $Q(s,a)$ arbitrarily for all $s \in \mathcal{S}$, $a \in \mathcal{A}(s)$
\For{each episode}
    \State Initialize $s$
    \For{each time step}
        \State Select $a_t$ using $\varepsilon$-greedy policy over $Q(s_t,\cdot)$
        \State Execute $a_t$, observe $r_{t+1}, s_{t+1}$
        \State Update $Q(s_t,a_t)$ using Eq.~\eqref{eq:q_learning_update}
        \State $s_t \gets s_{t+1}$
    \EndFor
\EndFor
\end{algorithmic}
\end{algorithm}

Although Q-Learning is guaranteed to converge under standard assumptions in tabular settings, its direct application becomes infeasible in large or continuous state spaces. This limitation motivated the use of function approximation, culminating in the development of Deep Q-Networks.

\subsection{Deep Q-Networks}
\label{sec:dqn}

Early attempts to combine \gls{rl} with function approximation include TD-Gammon~\cite{tesauro1994td}, which employed a neural network to approximate the value function within the TD($\lambda$) framework. Building upon these foundational ideas, \cite{mnih2015human} introduced the \gls{dqn}, a scalable framework that successfully integrates deep neural networks with Q-Learning.

DQN approximates the action-value function using a deep neural network parameterized by $\theta$ and introduces several key mechanisms to stabilize learning:

\begin{itemize}
    \item \textbf{Deep State Representation:}
    High-dimensional observations, such as images, are processed using Convolutional Neural Networks (CNNs)~\cite{lecun2010convolutional}, allowing the agent to learn compact feature representations directly from raw sensory input.
    
    \item \textbf{Experience Replay:}
    Transitions $(s,a,r,s')$ are stored in a replay buffer $\mathcal{D}$ and sampled uniformly during training. This breaks temporal correlations between samples and improves data efficiency.
    
    \item \textbf{Target Network:}
    A separate target network with parameters $\theta^-$ is used to compute TD targets. The parameters $\theta^-$ are updated periodically, reducing training instabilities caused by rapidly changing targets.
\end{itemize}

The DQN update rule follows the Q-Learning principle:
\begin{equation}
    \label{eq:dqn_update}
    Q(s_t, a_t; \theta) \leftarrow Q(s_t, a_t; \theta)
    + \alpha \left[
        r_{t+1}
        + \gamma \max_{a'} Q(s_{t+1}, a'; \theta^-)
        - Q(s_t, a_t; \theta)
    \right].
\end{equation}

Equivalently, learning can be interpreted as minimizing the squared TD error through the loss function:
\begin{equation}
    \mathcal{L}(\theta) =
    \mathbb{E}_{(s,a,r,s') \sim \mathcal{D}}
    \left[
        \left(
        r + \gamma \max_{a'} Q(s',a';\theta^-)
        - Q(s,a;\theta)
        \right)^2
    \right].
\end{equation}

Algorithm~\ref{alg:dqn} summarizes its training procedure.

\begin{algorithm}
\caption{Deep Q-Network}
\label{alg:dqn}
\begin{algorithmic}[1]
\Require Initialize Q-network parameters $\theta$
\Require Initialize target network parameters $\theta^- \leftarrow \theta$
\Require Initialize replay buffer $\mathcal{D}$
\For{each episode}
    \State Initialize $s$
    \For{each step}
        \State Select $a_t$ using $\varepsilon$-greedy policy over $Q(s_t,\cdot;\theta)$
        \State Execute $a_t$, observe $r_{t+1}, s_{t+1}$
        \State Store $(s_t,a_t,r_{t+1},s_{t+1})$ in $\mathcal{D}$
        \State Sample mini-batch from $\mathcal{D}$
        \State Perform gradient descent on $\mathcal{L}(\theta)$
        \State Every $C$ steps, update target network: $\theta^- \leftarrow \theta$
        \State $s_t \gets s_{t+1}$
    \EndFor
\EndFor
\end{algorithmic}
\end{algorithm}

\section{Policy Gradient Methods}
\label{sec:pg}

Policy Gradient (PG) methods constitute a class of reinforcement learning algorithms that represent the agent’s behavior directly through a parameterized policy, without requiring an explicit action-value function for action selection~\cite{sutton1999policy}. Let $\pi_\theta(a \mid s)$ denote a stochastic policy parameterized by $\theta \in \mathbb{R}^n$, which defines the probability of selecting action $a$ in state $s$:
\[
\pi(a \mid s, \theta) \doteq \Pr\{A_t = a \mid S_t = s, \theta_t = \theta\}.
\]

The objective of policy gradient methods is to find policy parameters $\theta$ that maximize a scalar performance measure $J(\theta)$, typically defined as the expected return obtained by following $\pi_\theta$. A common choice is:
\begin{equation}
    J(\theta) = \mathbb{E}_{\pi_\theta} \left[ \sum_{t=0}^{\infty} \gamma^t r_{t+1} \right],
\end{equation}
where the expectation is taken over trajectories induced by the policy $\pi_\theta$ and the environment dynamics.

Policy gradient methods perform \emph{gradient ascent} on this objective by iteratively updating the policy parameters in the direction of the estimated gradient:
\begin{equation}
\label{eq:gradient_ascent}
    \theta_{t+1} \doteq \theta_t + \alpha \, \widehat{\nabla_\theta J(\theta_t)},
\end{equation}
where $\alpha > 0$ is the step size and $\widehat{\nabla_\theta J(\theta_t)}$ is a stochastic estimate of the true gradient. Any algorithm that follows this update principle falls under the policy gradient framework.

The theoretical foundation underlying these methods is given by the \emph{Policy Gradient Theorem}~\cite{sutton1999policy}, which provides an explicit expression for the gradient of $J(\theta)$ without requiring differentiation of the environment dynamics.

\begin{theorem}[Policy Gradient Theorem]
\label{theo:pg_theorem}
\begin{gather}
\begin{split}
\nabla_\theta J(\pi_\theta)
&= \int_{\mathcal{S}} p^{\pi}(s)
    \int_{\mathcal{A}} \nabla_\theta \pi_\theta(a \mid s)
    Q_\pi(s,a) \, da \, ds \\
&= \mathbb{E}_{s \sim p^{\pi},\, a \sim \pi_\theta}
\left[
\nabla_\theta \log \pi_\theta(a \mid s)\, Q_\pi(s,a)
\right],
\end{split}
\label{eq:pg_theorem}
\end{gather}
\end{theorem}

In this expression, $p^{\pi}(s)$ denotes the discounted state visitation distribution induced by policy $\pi_\theta$, and $Q_\pi(s,a)$ is the action-value function under the same policy. The theorem shows that the gradient of the expected return can be computed by weighting the score function $\nabla_\theta \log \pi_\theta(a \mid s)$ by the corresponding action-value, enabling unbiased gradient estimates from sampled trajectories.

Although $Q_\pi(s,a)$ appears in the gradient expression, it is not required for action selection and is often approximated or replaced by alternative signals, such as advantage functions or learned critics. This leads naturally to actor--critic architectures, which combine policy gradient updates with value function approximation.


\subsubsection{Actor-Critic}
\label{sec:actor_critic}
In this work, we employ an \textbf{Actor-Critic} algorithm, which combines the strengths of \gls{pg} methods and value-based approaches. The \textbf{actor} represents the policy $\pi_\theta(a|s)$, updated using the gradient from Theorem~\ref{theo:pg_theorem}, while the \textbf{critic} estimates the action-value function $Q_\pi(s,a)$ through a parameterized function $Q_w(s,a)$. The critic provides feedback to the actor by evaluating its actions.

Figure~\ref{fig:actor_critic} illustrates the general training loop of an Actor-Critic architecture.

\begin{figure}[ht]
    \caption{Example of Actor-Critic training loop. The $Q$-values of the critic module are used to train the actor module.}
    \centering
    \includegraphics[width=0.75\textwidth]{images/fundamentacao/rl/actor_critic.png}
    \par\medskip\ABNTEXfontereduzida\selectfont\textbf{Source:} Author
    \label{fig:actor_critic}
\end{figure} 

The key distinction between Actor-Critic and standard \gls{pg} methods lies in the use of bootstrapping. Rather than relying on the true value $Q_\pi(s,a)$—which is often unknown—the critic substitutes it with a learned approximation $Q_w(s,a)$. This approximation introduces bias and its effectiveness depends on the quality of the function approximator.

However, under specific conditions, this bias can be eliminated. If the approximator satisfies $Q_w(s,a) = \nabla_\theta \log \pi_\theta(a|s)^T w$ and the parameters $w$ minimize the mean squared error:
\[
\epsilon^2(w) = \E_{s \sim p, a \sim \pi_\theta}\left[(Q_w(s,a) - Q_\pi(s,a))^2\right],
\]
then the update remains unbiased \cite{sutton2018reinforcement}.
\subsection{Deterministic Policy Gradient Methods}
\label{sec:dpg}

The \gls{dpg} algorithm was introduced as an \textbf{off-policy} Actor–Critic method capable of learning a deterministic target policy from data generated by a separate behavior policy \cite{dpg}. This formulation builds upon earlier off-policy policy gradient approaches, such as \cite{offpac}, in which a behavior policy $\beta(a \mid s)$—distinct from the target policy $\pi_{\theta}(a \mid s)$—is used to collect experience for policy updates.

In contrast to stochastic policy gradient methods, which learn a probability distribution over actions $\pi_\theta(a \mid s)$, \gls{dpg} maintains a parameterized deterministic policy $\mu(s \mid \theta_\mu)$ that directly maps each state to a single action. That is, instead of estimating action probabilities, the actor explicitly estimates the action to be executed in a given state, which motivates the term deterministic. Exploration is therefore handled externally, typically by injecting noise into the behavior policy used for data collection, rather than being encoded in the policy representation itself~\cite{ddpg}.

The critic, denoted $Q_w(s,a)$, approximates the action–value function and is trained using the Bellman equation (Equation~\ref{eq:bellman_value}). The learning objective of the actor is to maximize the expected return induced by the deterministic policy,
\begin{equation}
\label{eq:dpg_objective}
J(\theta_\mu) = \E_{s_t \sim p^\beta}\left[ Q_w\big(s_t, \mu(s_t \mid \theta_\mu)\big) \right],
\end{equation}
where $p^\beta$ denotes the state distribution induced by the behavior policy $\beta$.

Applying the deterministic policy gradient theorem, the gradient of this objective with respect to the actor parameters can be written as
\begin{gather}
\begin{split}
\label{eq:dpg}
\nabla_{\theta_\mu} J(\theta_\mu)
&\approx \E_{s_t \sim p^\beta}\left[\nabla_{\theta_\mu} Q_w(s,a)\big|{s=s_t,, a=\mu(s_t)}\right] \\
&= \E{s_t \sim p^\beta}\left[\nabla_{a} Q_w(s,a)\big|{s=s_t,, a=\mu(s_t)}
\cdot \nabla{\theta_\mu} \mu(s \mid \theta_\mu)\big|_{s=s_t}\right].
\end{split}
\end{gather}

While \gls{dpg} is effective in continuous action spaces, its performance is often limited by the function approximator’s capacity to estimate $Q$ accurately—especially in high-dimensional state spaces. To address this, \cite{ddpg} proposed the \gls{ddpg} algorithm, inspired by the \gls{dqn} architecture. \gls{ddpg} incorporates deep neural networks to approximate both the actor and the critic, and leverages Experience Replay for sample efficiency and stability.

In addition to these features, \gls{ddpg} introduces several techniques to stabilize training:

- \textbf{Target Networks:} As in \gls{dqn} \cite{dqnTarget}, separate target networks are used for both the actor and critic to prevent harmful feedback loops. These target networks are updated slowly using a soft update rule: 
  \[
  \theta' \gets \tau\theta + (1 - \tau)\theta', \quad 0 < \tau \ll 1,
  \]
  where $\theta'$ are the target weights, and $\theta$ are the main network’s weights.

- \textbf{Exploration Noise:} Since \gls{ddpg} learns a deterministic policy, it requires an external mechanism for exploration. The algorithm employs an Ornstein-Uhlenbeck process \cite{uhlenbeck1930theory} to generate temporally correlated noise, which is added to the actions during training. 
\subsection{Soft Actor-Critic}
\label{sec:sac}

The \gls{sac} algorithm extends the deterministic actor-critic approach by addressing some of the limitations of \gls{ddpg}, such as poor exploration and training instability \cite{sac}. While \gls{ddpg} uses a deterministic policy and requires carefully tuned exploration noise, \gls{sac} employs a \emph{stochastic policy} and introduces the concept of \emph{entropy maximization} as part of the optimization objective.

The key idea behind \gls{sac} is to encourage exploration by maximizing a trade-off between expected return and the entropy of the policy. Formally, the objective becomes:

\begin{equation}
\label{eq:sac_objective}
J(\pi) = \sum_{t=0}^{T} \E_{(s_t,a_t)\sim \rho_\pi} \left[ r(s_t,a_t) + \alpha \mathcal{H}(\pi(\cdot|s_t)) \right],
\end{equation}

\noindent
where $\mathcal{H}(\pi(\cdot|s_t)) = \E_{a_t \sim \pi}[-\log \pi(a_t|s_t)]$ is the entropy of the policy at state $s_t$, and $\alpha$ is a temperature parameter that controls the relative importance of the entropy term.

By encouraging higher entropy (i.e., more randomness), SAC ensures better exploration and robustness during training. Unlike \gls{ddpg}, \gls{sac} can learn from a wider range of trajectories without collapsing to suboptimal deterministic behaviors.

\gls{sac} introduces several innovations over \gls{ddpg}:

\begin{itemize}
    \item \textbf{Stochastic Policies:} Instead of outputting a single deterministic action, SAC learns a policy $\pi(a|s)$ that outputs a distribution over actions, typically modeled by a squashed Gaussian.
    
    \item \textbf{Entropy-Regularized Objective:} As seen in Equation~\ref{eq:sac_objective}, SAC maximizes both expected reward and entropy, making the agent more exploratory and less likely to get stuck in poor local optima.
    
    \item \textbf{Twin Q-networks:} \gls{sac} uses two critic networks to mitigate overestimation bias. The minimum of the two Q-values is used in the target computation.
    
    \item \textbf{Automatic Temperature Adjustment:} \gls{sac} can learn the temperature parameter $\alpha$ automatically by adjusting it to match a target entropy level. This removes the need for manual tuning of exploration-exploitation trade-offs.
\end{itemize}

These enhancements make \gls{sac} highly sample-efficient and robust in continuous control tasks. It has shown strong empirical performance across a variety of benchmark environments, often outperforming \gls{ddpg} in both sample efficiency and final performance.

\begin{algorithm}
    \caption{Soft Actor-Critic (SAC) Algorithm}
    \begin{algorithmic}
        \State Initialize actor parameters $\theta$, critic parameters $w_1$, $w_2$, temperature parameter $\alpha$
        \State Initialize target critic parameters $w_1' \gets w_1$, $w_2' \gets w_2$
        \State Initialize empty replay buffer $\mathcal{D}$
        \For{episode = 1 to M}
            \State Receive initial state $s_0$
            \For{$t = 1$ to T}
                \State Sample action $a_t \sim \pi_\theta(\cdot|s_t)$
                \State Execute action $a_t$, observe reward $r_t$ and next state $s_{t+1}$
                \State Store transition $(s_t, a_t, r_t, s_{t+1})$ in $\mathcal{D}$
                
                \State Sample minibatch $\{(s_i, a_i, r_i, s_{i+1})\}_{i=1}^N$ from $\mathcal{D}$
                \State Sample $a_{i+1} \sim \pi_\theta(\cdot|s_{i+1})$ and compute $\log \pi_\theta(a_{i+1}|s_{i+1})$
                \State $y_i \gets r_i + \gamma \left( \min_{j=1,2} Q_{w_j'}(s_{i+1}, a_{i+1}) - \alpha \log \pi_\theta(a_{i+1}|s_{i+1}) \right)$
                
                \State Update critics by minimizing:
                \[
                L(w_j) = \frac{1}{N} \sum_i (Q_{w_j}(s_i, a_i) - y_i)^2, \quad \text{for } j = 1, 2
                \]
                
                \State Update actor by minimizing:
                \[
                J(\theta) = \frac{1}{N} \sum_i \left( \alpha \log \pi_\theta(a_i|s_i) - \min_{j=1,2} Q_{w_j}(s_i, a_i) \right)
                \]

                \State Update temperature (optional, if $\alpha$ is learned):
                \[
                J(\alpha) = \frac{1}{N} \sum_i \alpha \left( -\log \pi_\theta(a_i|s_i) - \mathcal{H}_{\text{target}} \right)
                \]

                \State Soft update target networks:
                \[
                w_j' \gets \tau w_j + (1 - \tau) w_j', \quad \text{for } j = 1, 2
                \]
            \EndFor
        \EndFor
    \end{algorithmic}
    \label{algo:sac}
\end{algorithm}

The methods reviewed in this chapter differ in their learning dynamics but share a fundamental reliance on the reward signal as the central driver of optimization. In this sense, reward design and prioritization constitute a common challenge across all frameworks. DyLam is designed as an algorithm-agnostic mechanism and can be combined with each of these methods without modifying their underlying update rules. By decoupling reward prioritization from the learning algorithm, DyLam provides an unified and principled way to handle decomposed reward structures across different Reinforcement Learning paradigms.
% \subsection{Farama Gymnasium}
\label{sec:gym}

The \textbf{Farama Gymnasium} is a standardized Python library for developing and interacting with reinforcement learning environments~\cite{towers2024gymnasium}. It is the spiritual successor to OpenAI's Gym, which has been a cornerstone for the RL community since its release. With the deprecation of the original Gym repository, the Farama Foundation took the responsibility of maintaining and extending its functionality under the name Gymnasium\footnote{\url{https://github.com/Farama-Foundation/Gymnasium}}. This pattern has become ubiquitous in RL workflows, enabling efficient testing, benchmarking, and integration across tools~\cite{mnih2015human, dqnTarget, ddpg, sac}.

The core idea behind Gymnasium is to provide a unified API for RL environments, enabling researchers and developers to experiment with a wide range of tasks through a common interface. This standardization facilitates reproducibility, benchmarking, and interoperability between agents, environments, and libraries.

\subsubsection{The Environment Interface Pattern}
Gymnasium enforces a clear abstraction separating the \emph{environment} from the \emph{simulator}. The environment handles the agent-environment loop, while the simulator (e.g., physics engines or emulators) encapsulates the dynamics and rendering. This abstraction allows a consistent API regardless of the domain—be it gridworlds~\cite{MinigridMiniworld23}, robotics~\cite{todorov2012mujoco}, or Atari games~\cite{bellemare13arcade}.

To define a new environment in Gymnasium, several components must be specified:

\begin{itemize}
    \item \textbf{Observation space:} Defines the structure and type of states returned to the agent. This could be discrete labels, continuous vectors, images, etc.
    \item \textbf{Action space:} Specifies the valid actions an agent can take. Gymnasium supports discrete, continuous, and compound spaces.
    \item \textbf{Reward signal:} A scalar or vector indicating the immediate feedback for the agent’s action. The semantics are domain-specific but must follow the Gymnasium interface.
    \item \textbf{Core methods:}
    \begin{itemize}
        \item \texttt{reset()}: Initializes the environment to a starting state, returning the initial observation.
        \item \texttt{step(action)}: Advances the environment one timestep given the action. It returns a tuple of $(\text{observation}, \text{reward}, \text{terminated}, \text{truncated}, \text{info})$.
    \end{itemize}
\end{itemize}

\subsubsection{Registration and Wrappers}
Gymnasium uses a registry system to manage available environments, allowing users to instantiate them using a string identifier. Additionally, it provides \emph{wrappers}, which are modular tools for modifying environments (e.g., observation normalization, reward shaping, or frame skipping) without changing their core logic.

\subsubsection{Basic Usage Example}
The following Python snippet demonstrates the basic usage of a Gymnasium environment using the \texttt{LunarLander-v3} domain. It illustrates environment creation, stepping through the simulation, and resetting after episode termination:

\begin{lstlisting}[language=Python, caption=Basic usage of Gymnasium, label=code:gym]
import gymnasium as gym

# Initialise the environment
env = gym.make("LunarLander-v3", render_mode="human")

# Reset the environment to generate the first observation
observation, info = env.reset(seed=42)
for _ in range(1000):
    # This is where you would insert your policy
    action = env.action_space.sample()

    # Step through the environment with the action
    observation, reward, terminated, truncated, info = env.step(action)

    # If the episode has ended, reset to start a new episode
    if terminated or truncated:
        observation, info = env.reset()

env.close()
\end{lstlisting}


