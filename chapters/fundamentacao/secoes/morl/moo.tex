\subsection{Multi-Objective Optimization}
\label{sec:moo}

\gls{moo} is a subfield of multi-criteria decision making in which an agent considers multiple objectives when making a decision. For instance, consider a scenario where a robot must navigate to a target point $(x, y)$ while being efficient in terms of both energy consumption and travel time. In this case, the robot faces three potentially conflicting objectives: path planning, energy efficiency, and time minimization. In simple environments, optimizing for the shortest path may also minimize energy and time. However, in more complex environments—such as those with dynamic obstacles—minimizing travel time may require taking a longer or more energy-consuming path.

In optimization, candidate solutions reside in the \textbf{decision space}, and the performance of each solution is evaluated by a set of \textbf{objective functions}~\cite{emmerich2018tutorial}. These concepts resemble the action space $A$ and reward functions $r(s,a)$ in reinforcement learning, as objective functions are typically derived from either analytical models or empirical evaluations. The goal in multi-objective settings is to find solutions that appropriately balance all objectives.

A central concept in \gls{moo} is the existence of trade-offs among objectives. These trade-offs, known as \textbf{Pareto conflicts}, occur when improving one objective necessarily leads to the deterioration of another. A solution is said to \textbf{Pareto dominate} another if it is not worse in any objective and strictly better in at least one~\cite{ehrgott2012vilfredo}. The set of solutions that are not dominated by any other composes the \textbf{Pareto Front}, which represents the boundary between dominated and non-dominated solutions, as illustrated in \fref{fig:fundamentacao/pareto_front}.

Notably, in \fref{fig:fundamentacao/pareto_front}, three fundamental concepts are depicted—two explicitly and one implicitly. First, the \textbf{reference point} establishes a baseline by defining the worst possible solution within the objective space. Second, the \textbf{hypervolume} quantifies the space dominated by the Pareto front relative to this reference point, providing a numerical indicator of performance. This metric becomes particularly relevant in high-dimensional objective spaces, where direct visualization of the \gls{pf} is infeasible. The implicit concept is the \textbf{cardinality} of the front, which, while not directly implying superiority between methods, offers insights into the diversity and distribution of non-dominated solutions.

\begin{figure}
\caption{Illustration of the Pareto front in a bi-objective maximization problem. 
Non-dominated solutions (Pareto-optimal) form the frontier, while dominated solutions lie below it. 
The red cross marks the \textbf{reference point}, representing the worst possible objective values used to define a performance baseline. 
The shaded area denotes the \textbf{hypervolume}, which measures the portion of the objective space dominated by the Pareto front relative to the reference point. 
Each point represents a potential trade-off between the two objectives.}
    \centering
    \includegraphics[width=\linewidth]{images/fundamentacao/morl/pareto.png}
    \label{fig:fundamentacao/pareto_front}
    \par\medskip\ABNTEXfontereduzida\selectfont\textbf{Source:} Author
\end{figure}

\begin{definition}
\label{def:moo}
\textbf{Multi-Objective Optimization.} Given $m > 1$ objective functions $f_1: X \rightarrow \mathbb{R}, \ldots, f_m: X \rightarrow \mathbb{R}$ that map a decision space $X$ into $\mathbb{R}^m$, a multi-objective optimization problem is defined as:
\begin{equation}
\label{eq:moo}
\text{minimize } f_1(\mathbf{x}), \ldots, \text{minimize } f_m(\mathbf{x}), \quad \mathbf{x} \in X
\end{equation}
The goal is to find solutions $\mathbf{x}$ that appropriately balance the objectives, rather than optimizing a single one in isolation.
\end{definition}

\begin{definition}
\label{def:pareto_dominance}
\textbf{Pareto Dominance.} Given two solutions $\mathbf{x}, \mathbf{y} \in X$ and $m$ objective functions $f_1, \ldots, f_m$, we say that $\mathbf{x}$ \emph{Pareto dominates} $\mathbf{y}$ (denoted $\mathbf{x} \prec \mathbf{y}$) if and only if:
\begin{equation}
\label{eq:pareto_dominance}
\forall i \in \{1, \ldots, m\},\ f_i(\mathbf{x}) \leq f_i(\mathbf{y}) \quad \text{and} \quad \exists j \in \{1, \ldots, m\},\ f_j(\mathbf{x}) < f_j(\mathbf{y}).
\end{equation}
\end{definition}

\begin{definition}
\label{def:pareto_conflict}
\textbf{Pareto Conflict.} A \emph{Pareto conflict} exists between two or more objectives when improvement in one objective necessarily leads to degradation in at least one other objective for all feasible solutions. That is, no single solution simultaneously optimizes all objectives without trade-offs.
\end{definition}

\begin{definition}
\label{def:pareto_front}
\textbf{Pareto Front.} The \emph{Pareto front} (or Pareto frontier) is the set of all \emph{non-dominated} solutions in the decision space $X$. Formally, it is defined as:
\begin{equation}
\label{eq:pareto_front}
\mathcal{P} = \left\{ \mathbf{x} \in X \ \middle| \ \nexists\ \mathbf{y} \in X \text{ such that } \mathbf{y} \prec \mathbf{x} \right\}.
\end{equation}
Each solution $\mathbf{x} \in \mathcal{P}$ represents a distinct trade-off between objectives, and cannot be improved in one objective without worsening at least one other.
\end{definition}

