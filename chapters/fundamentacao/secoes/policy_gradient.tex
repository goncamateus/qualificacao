\section{Policy Gradient Methods}
\label{sec:pg}

Policy Gradient (PG) methods constitute a class of reinforcement learning algorithms that represent the agent’s behavior directly through a parameterized policy, without requiring an explicit action-value function for action selection~\cite{sutton1999policy}. Let $\pi_\theta(a \mid s)$ denote a stochastic policy parameterized by $\theta \in \mathbb{R}^n$, which defines the probability of selecting action $a$ in state $s$:
\[
\pi(a \mid s, \theta) \doteq \Pr\{A_t = a \mid S_t = s, \theta_t = \theta\}.
\]

The objective of policy gradient methods is to find policy parameters $\theta$ that maximize a scalar performance measure $J(\theta)$, typically defined as the expected return obtained by following $\pi_\theta$. A common choice is:
\begin{equation}
    J(\theta) = \mathbb{E}_{\pi_\theta} \left[ \sum_{t=0}^{\infty} \gamma^t r_{t+1} \right],
\end{equation}
where the expectation is taken over trajectories induced by the policy $\pi_\theta$ and the environment dynamics.

Policy gradient methods perform \emph{gradient ascent} on this objective by iteratively updating the policy parameters in the direction of the estimated gradient:
\begin{equation}
\label{eq:gradient_ascent}
    \theta_{t+1} \doteq \theta_t + \alpha \, \widehat{\nabla_\theta J(\theta_t)},
\end{equation}
where $\alpha > 0$ is the step size and $\widehat{\nabla_\theta J(\theta_t)}$ is a stochastic estimate of the true gradient. Any algorithm that follows this update principle falls under the policy gradient framework.

The theoretical foundation underlying these methods is given by the \emph{Policy Gradient Theorem}~\cite{sutton1999policy}, which provides an explicit expression for the gradient of $J(\theta)$ without requiring differentiation of the environment dynamics.

\begin{theorem}[Policy Gradient Theorem]
\label{theo:pg_theorem}
\begin{gather}
\begin{split}
\nabla_\theta J(\pi_\theta)
&= \int_{\mathcal{S}} p^{\pi}(s)
    \int_{\mathcal{A}} \nabla_\theta \pi_\theta(a \mid s)
    Q_\pi(s,a) \, da \, ds \\
&= \mathbb{E}_{s \sim p^{\pi},\, a \sim \pi_\theta}
\left[
\nabla_\theta \log \pi_\theta(a \mid s)\, Q_\pi(s,a)
\right],
\end{split}
\label{eq:pg_theorem}
\end{gather}
\end{theorem}

In this expression, $p^{\pi}(s)$ denotes the discounted state visitation distribution induced by policy $\pi_\theta$, and $Q_\pi(s,a)$ is the action-value function under the same policy. The theorem shows that the gradient of the expected return can be computed by weighting the score function $\nabla_\theta \log \pi_\theta(a \mid s)$ by the corresponding action-value, enabling unbiased gradient estimates from sampled trajectories.

Although $Q_\pi(s,a)$ appears in the gradient expression, it is not required for action selection and is often approximated or replaced by alternative signals, such as advantage functions or learned critics. This leads naturally to actor--critic architectures, which combine policy gradient updates with value function approximation.


\subsubsection{Actor-Critic}
\label{sec:actor_critic}
In this work, we employ an \textbf{Actor-Critic} algorithm, which combines the strengths of \gls{pg} methods and value-based approaches. The \textbf{actor} represents the policy $\pi_\theta(a|s)$, updated using the gradient from Theorem~\ref{theo:pg_theorem}, while the \textbf{critic} estimates the action-value function $Q_\pi(s,a)$ through a parameterized function $Q_w(s,a)$. The critic provides feedback to the actor by evaluating its actions.

Figure~\ref{fig:actor_critic} illustrates the general training loop of an Actor-Critic architecture.

\begin{figure}[ht]
    \caption{Example of Actor-Critic training loop. The $Q$-values of the critic module are used to train the actor module.}
    \centering
    \includegraphics[width=0.75\textwidth]{images/fundamentacao/rl/actor_critic.png}
    \par\medskip\ABNTEXfontereduzida\selectfont\textbf{Source:} Author
    \label{fig:actor_critic}
\end{figure} 

The key distinction between Actor-Critic and standard \gls{pg} methods lies in the use of bootstrapping. Rather than relying on the true value $Q_\pi(s,a)$—which is often unknown—the critic substitutes it with a learned approximation $Q_w(s,a)$. This approximation introduces bias and its effectiveness depends on the quality of the function approximator.

However, under specific conditions, this bias can be eliminated. If the approximator satisfies $Q_w(s,a) = \nabla_\theta \log \pi_\theta(a|s)^T w$ and the parameters $w$ minimize the mean squared error:
\[
\epsilon^2(w) = \E_{s \sim p, a \sim \pi_\theta}\left[(Q_w(s,a) - Q_\pi(s,a))^2\right],
\]
then the update remains unbiased \cite{sutton2018reinforcement}.
\subsection{Deterministic Policy Gradient Methods}
\label{sec:dpg}

The \gls{dpg} algorithm was introduced as an \textbf{off-policy} Actor–Critic method capable of learning a deterministic target policy from data generated by a separate behavior policy \cite{dpg}. This formulation builds upon earlier off-policy policy gradient approaches, such as \cite{offpac}, in which a behavior policy $\beta(a \mid s)$—distinct from the target policy $\pi_{\theta}(a \mid s)$—is used to collect experience for policy updates.

In contrast to stochastic policy gradient methods, which learn a probability distribution over actions $\pi_\theta(a \mid s)$, \gls{dpg} maintains a parameterized deterministic policy $\mu(s \mid \theta_\mu)$ that directly maps each state to a single action. That is, instead of estimating action probabilities, the actor explicitly estimates the action to be executed in a given state, which motivates the term deterministic. Exploration is therefore handled externally, typically by injecting noise into the behavior policy used for data collection, rather than being encoded in the policy representation itself~\cite{ddpg}.

The critic, denoted $Q_w(s,a)$, approximates the action–value function and is trained using the Bellman equation (Equation~\ref{eq:bellman_value}). The learning objective of the actor is to maximize the expected return induced by the deterministic policy,
\begin{equation}
\label{eq:dpg_objective}
J(\theta_\mu) = \E_{s_t \sim p^\beta}\left[ Q_w\big(s_t, \mu(s_t \mid \theta_\mu)\big) \right],
\end{equation}
where $p^\beta$ denotes the state distribution induced by the behavior policy $\beta$.

Applying the deterministic policy gradient theorem, the gradient of this objective with respect to the actor parameters can be written as
\begin{gather}
\begin{split}
\label{eq:dpg}
\nabla_{\theta_\mu} J(\theta_\mu)
&\approx \E_{s_t \sim p^\beta}\left[\nabla_{\theta_\mu} Q_w(s,a)\big|{s=s_t,, a=\mu(s_t)}\right] \\
&= \E{s_t \sim p^\beta}\left[\nabla_{a} Q_w(s,a)\big|{s=s_t,, a=\mu(s_t)}
\cdot \nabla{\theta_\mu} \mu(s \mid \theta_\mu)\big|_{s=s_t}\right].
\end{split}
\end{gather}

While \gls{dpg} is effective in continuous action spaces, its performance is often limited by the function approximator’s capacity to estimate $Q$ accurately—especially in high-dimensional state spaces. To address this, \cite{ddpg} proposed the \gls{ddpg} algorithm, inspired by the \gls{dqn} architecture. \gls{ddpg} incorporates deep neural networks to approximate both the actor and the critic, and leverages Experience Replay for sample efficiency and stability.

In addition to these features, \gls{ddpg} introduces several techniques to stabilize training:

- \textbf{Target Networks:} As in \gls{dqn} \cite{dqnTarget}, separate target networks are used for both the actor and critic to prevent harmful feedback loops. These target networks are updated slowly using a soft update rule: 
  \[
  \theta' \gets \tau\theta + (1 - \tau)\theta', \quad 0 < \tau \ll 1,
  \]
  where $\theta'$ are the target weights, and $\theta$ are the main network’s weights.

- \textbf{Exploration Noise:} Since \gls{ddpg} learns a deterministic policy, it requires an external mechanism for exploration. The algorithm employs an Ornstein-Uhlenbeck process \cite{uhlenbeck1930theory} to generate temporally correlated noise, which is added to the actions during training. 
\subsection{Soft Actor-Critic}
\label{sec:sac}

The \gls{sac} algorithm extends the deterministic actor-critic approach by addressing some of the limitations of \gls{ddpg}, such as poor exploration and training instability \cite{sac}. While \gls{ddpg} uses a deterministic policy and requires carefully tuned exploration noise, \gls{sac} employs a \emph{stochastic policy} and introduces the concept of \emph{entropy maximization} as part of the optimization objective.

The key idea behind \gls{sac} is to encourage exploration by maximizing a trade-off between expected return and the entropy of the policy. Formally, the objective becomes:

\begin{equation}
\label{eq:sac_objective}
J(\pi) = \sum_{t=0}^{T} \E_{(s_t,a_t)\sim \rho_\pi} \left[ r(s_t,a_t) + \alpha \mathcal{H}(\pi(\cdot|s_t)) \right],
\end{equation}

\noindent
where $\mathcal{H}(\pi(\cdot|s_t)) = \E_{a_t \sim \pi}[-\log \pi(a_t|s_t)]$ is the entropy of the policy at state $s_t$, and $\alpha$ is a temperature parameter that controls the relative importance of the entropy term.

By encouraging higher entropy (i.e., more randomness), SAC ensures better exploration and robustness during training. Unlike \gls{ddpg}, \gls{sac} can learn from a wider range of trajectories without collapsing to suboptimal deterministic behaviors.

\gls{sac} introduces several innovations over \gls{ddpg}:

\begin{itemize}
    \item \textbf{Stochastic Policies:} Instead of outputting a single deterministic action, SAC learns a policy $\pi(a|s)$ that outputs a distribution over actions, typically modeled by a squashed Gaussian.
    
    \item \textbf{Entropy-Regularized Objective:} As seen in Equation~\ref{eq:sac_objective}, SAC maximizes both expected reward and entropy, making the agent more exploratory and less likely to get stuck in poor local optima.
    
    \item \textbf{Twin Q-networks:} \gls{sac} uses two critic networks to mitigate overestimation bias. The minimum of the two Q-values is used in the target computation.
    
    \item \textbf{Automatic Temperature Adjustment:} \gls{sac} can learn the temperature parameter $\alpha$ automatically by adjusting it to match a target entropy level. This removes the need for manual tuning of exploration-exploitation trade-offs.
\end{itemize}

These enhancements make \gls{sac} highly sample-efficient and robust in continuous control tasks. It has shown strong empirical performance across a variety of benchmark environments, often outperforming \gls{ddpg} in both sample efficiency and final performance.

\begin{algorithm}
    \caption{Soft Actor-Critic (SAC) Algorithm}
    \begin{algorithmic}
        \State Initialize actor parameters $\theta$, critic parameters $w_1$, $w_2$, temperature parameter $\alpha$
        \State Initialize target critic parameters $w_1' \gets w_1$, $w_2' \gets w_2$
        \State Initialize empty replay buffer $\mathcal{D}$
        \For{episode = 1 to M}
            \State Receive initial state $s_0$
            \For{$t = 1$ to T}
                \State Sample action $a_t \sim \pi_\theta(\cdot|s_t)$
                \State Execute action $a_t$, observe reward $r_t$ and next state $s_{t+1}$
                \State Store transition $(s_t, a_t, r_t, s_{t+1})$ in $\mathcal{D}$
                
                \State Sample minibatch $\{(s_i, a_i, r_i, s_{i+1})\}_{i=1}^N$ from $\mathcal{D}$
                \State Sample $a_{i+1} \sim \pi_\theta(\cdot|s_{i+1})$ and compute $\log \pi_\theta(a_{i+1}|s_{i+1})$
                \State $y_i \gets r_i + \gamma \left( \min_{j=1,2} Q_{w_j'}(s_{i+1}, a_{i+1}) - \alpha \log \pi_\theta(a_{i+1}|s_{i+1}) \right)$
                
                \State Update critics by minimizing:
                \[
                L(w_j) = \frac{1}{N} \sum_i (Q_{w_j}(s_i, a_i) - y_i)^2, \quad \text{for } j = 1, 2
                \]
                
                \State Update actor by minimizing:
                \[
                J(\theta) = \frac{1}{N} \sum_i \left( \alpha \log \pi_\theta(a_i|s_i) - \min_{j=1,2} Q_{w_j}(s_i, a_i) \right)
                \]

                \State Update temperature (optional, if $\alpha$ is learned):
                \[
                J(\alpha) = \frac{1}{N} \sum_i \alpha \left( -\log \pi_\theta(a_i|s_i) - \mathcal{H}_{\text{target}} \right)
                \]

                \State Soft update target networks:
                \[
                w_j' \gets \tau w_j + (1 - \tau) w_j', \quad \text{for } j = 1, 2
                \]
            \EndFor
        \EndFor
    \end{algorithmic}
    \label{algo:sac}
\end{algorithm}

The methods reviewed in this chapter differ in their learning dynamics but share a fundamental reliance on the reward signal as the central driver of optimization. In this sense, reward design and prioritization constitute a common challenge across all frameworks. DyLam is designed as an algorithm-agnostic mechanism and can be combined with each of these methods without modifying their underlying update rules. By decoupling reward prioritization from the learning algorithm, DyLam provides an unified and principled way to handle decomposed reward structures across different Reinforcement Learning paradigms.