\section{Reinforcement Learning Benchmarks}
\label{sec:environments}

In previous chapters, we discussed the relationship between the proposed \gls{rl} framework and both multi-objective decomposition-based learning and traditional single-objective \gls{rl}. Although all environments considered in this work can be formulated within a multi-objective framework, they serve distinct experimental purposes. 

Some environments are explicitly designed to emphasize the construction and analysis of the Pareto front, focusing on the agent’s ability to discover a diverse set of trade-off solutions. Other environments are employed to evaluate the learning dynamics and sample efficiency of the proposed algorithm when dealing with multiple objectives under limited interaction budgets or data constraints. This distinction enables a comprehensive assessment of DyLam from complementary perspectives: one highlighting its capacity to approximate Pareto-optimal solution sets, and another emphasizing its learning efficiency in practical, data-driven scenarios.

\subsection{Pareto-Oriented Setups}
\label{sec:pareto_oriented_setups}

As discussed above, we aim to analyze DyLam’s ability to explore both the Pareto front and the associated weight space. To this end, we consider two well-established environments in the \gls{morl} literature~\cite{abels2019dynamicweightsmultiobjectivedeep, alegre2023sample, felten_toolkit_2023}.

\subsubsection{Multi-Objective HalfCheetah}

The Multi-Objective HalfCheetah (MO-HalfCheetah) environment is a multi-objective extension of the original HalfCheetah task from the MuJoCo benchmark suite~\cite{todorov2012mujoco}. It is designed for continuous robotic control and is based on a two-dimensional articulated robot composed of nine rigid body segments connected by eight joints. The primary control objective is to apply torques to the robot’s joints in order to propel the agent forward (to the right) as efficiently and rapidly as possible. A visual representation of the environment is shown in \fref{fig:envs/mo_halfcheetah}.

\begin{figure}[ht]
    \centering
    \caption{Multi-Objective HalfCheetah environment. The agent controls a planar articulated robot and must balance forward velocity and energy efficiency objectives.}
    \includegraphics[width=.5\textwidth]{images/environments/halfcheetah.png}
    \par\medskip\ABNTEXfontereduzida\selectfont\textbf{Source:} Author
    \label{fig:envs/mo_halfcheetah}
\end{figure}

The MO-HalfCheetah environment is formally characterized by a continuous action space, a continuous observation space, and a vector-valued reward function that explicitly captures multiple, potentially conflicting objectives.

\paragraph{Multi-Objective Reward Function.}
In contrast to the single-objective formulation, MO-HalfCheetah defines a two-dimensional reward vector $\vec{r}_t \in \mathbb{R}^2$:
\begin{enumerate}
    \item \textbf{Forward velocity reward}, proportional to the agent’s forward displacement, with $R_{max}=800$ and $R_{min}=0$.
    \item \textbf{Control cost}, a negative reward proportional to the squared magnitude of the applied torques, penalizing excessive energy expenditure, with $R_{max}=-200$ and $R_{min}=-800$.
\end{enumerate}

In the original \texttt{HalfCheetah-v5} environment, these objectives are combined through linear scalarization,
\begin{equation}
r_t = r_t^{\text{forward}} - 0.1 \cdot r_t^{\text{control}}.
\end{equation}

However, in the DyLam experiments, this scalarization is removed, and the objectives are treated independently to allow explicit exploration of the Pareto front.

The MO-HalfCheetah environment does not define terminal conditions based on success or failure states. Episodes are truncated after a fixed horizon of 1000 time steps. In this work, we additionally consider this benchmark as a Learning-Dynamics-Oriented Setup using the distance traveled from the origin as a final performance indicator.

The HalfCheetah task is instantiated using the \texttt{HalfCheetah-v2} version and is employed as one of several continuous-action benchmark problems commonly used to evaluate multi-objective reinforcement learning algorithms~\cite{pgmorl, alegre2023sample, felten2024multi}. 



\subsubsection{Minecart}

The Multi-Objective Minecart (MO-Minecart) environment is a discrete multi-objective benchmark originally introduced in~\cite{abels2019dynamicweightsmultiobjectivedeep}. Since then, it has become a widely adopted testbed for evaluating multi-objective reinforcement learning algorithms~\cite{felten_toolkit_2023}, particularly due to its clear and interpretable trade-offs between competing reward components. An overview of the environment layout is illustrated in \fref{fig:envs/minecart}.

\begin{figure}[ht]
    \centering
    \caption{Minecart environment. The agent navigates a 2D space to collect multiple resource types under fuel constraints, inducing conflicting objectives.}
    \includegraphics[width=.5\textwidth]{images/environments/minecart.png}
    \par\medskip\ABNTEXfontereduzida\selectfont\textbf{Source:} Author
    \label{fig:envs/minecart}
\end{figure}

In Minecart, the agent controls a mining cart operating in a two-dimensional continuous space populated with multiple mines, each associated with a distinct resource type. The agent’s objective is to navigate the environment, collect resources from the mines, and return them to a central depot for processing. The episode terminates once the agent exhausts its fuel budget.

The defining characteristic of the Minecart environment is its inherently conflicting reward structure. Each resource type corresponds to a separate objective, and collecting a greater amount of one resource typically limits the agent’s ability to collect others due to fuel constraints and spatial layout. As a result, the learning problem naturally induces a Pareto front representing different trade-offs among resource collection strategies.

\paragraph{Multi-Objective Reward Function.}
The reward at each episode is represented by a vector $\vec{r} \in \mathbb{R}^3$, where each dimension corresponds to the cumulative amount of a specific resource successfully delivered to the depot and fuel spent. The maximum and minimum cumulative reward for each component is described by $\vec{R}_{max}=\{1.5, 1.5, 0\}$ and $\vec{R}_{min}=\{0, 0, -20\}$, respectively.

This explicit decomposition of objectives makes Minecart particularly suitable for studying adaptive weighting mechanisms and Pareto front exploration. In contrast to MO-HalfCheetah, Minecart emphasizes long-horizon planning and strategic trade-offs under resource constraints rather than fine-grained motor control.

Due to these properties, Minecart has been extensively used in the MORL literature as a benchmark for evaluating algorithms that aim to balance competing objectives and to recover diverse sets of Pareto-optimal policies~\cite{abels2019dynamicweightsmultiobjectivedeep, alegre2023sample, felten2024multi}. In this work, it serves as a complementary Pareto-oriented environment, enabling a clear analysis of DyLam’s ability to adaptively prioritize objectives and explore the Pareto front in a structured and interpretable setting.

\subsection{Learning-Dynamics-Oriented Setups}
\label{sec:traditional_setups}

In contrast to the Pareto-oriented environments described previously, the following setups are employed to evaluate DyLam’s learning dynamics under constrained interaction budgets and limited data availability. Rather than emphasizing the explicit construction of Pareto fronts, these environments are designed to assess how effectively the proposed method prioritizes reward components during training and how rapidly it converges to task-relevant behaviors.

To this end, we consider two environments with distinct characteristics: a discrete, symbolic domain (Taxi-v3) and a continuous, physics-based robotic task (VSS-v0). In both cases, the original reward functions are decomposed into multiple components to enable the application of DyLam and to facilitate a detailed analysis of reward prioritization throughout training.

\subsubsection{Taxi-v3}

Taxi-v3 is a classic discrete benchmark environment from the Gymnasium \textit{toy\_text} suite~\cite{towers2024gymnasium}. The task consists of a taxi navigating a grid-world to pick up a passenger at a designated location and safely deliver them to a target destination. The environment features sparse rewards, strict action constraints, and penalties for illegal actions, making it well suited for studying learning efficiency and credit assignment. A schematic representation of the environment is shown in \fref{fig:envs/taxi}.

\begin{figure}[ht]
    \centering
    \caption{Taxi-v3 environment. The agent navigates a discrete grid to pick up and drop off a passenger while avoiding illegal actions.}
    \includegraphics[width=.5\textwidth]{images/environments/taxi.png}
    \par\medskip\ABNTEXfontereduzida\selectfont\textbf{Source:} Author
    \label{fig:envs/taxi}
\end{figure}

To enable multi-objective learning, we decompose the original scalar reward into three distinct components:
\begin{enumerate}
    \item \textbf{Energy}: a negative reward associated with each movement action, encouraging shorter and more efficient trajectories.
    \item \textbf{Passenger drop-off}: a positive reward granted upon successfully delivering the passenger to the correct destination.
    \item \textbf{Illegal action}: a penalty applied when the agent attempts an invalid pickup or drop-off action.
\end{enumerate}

The reward bounds for this environment are defined as
\[
\vec{R}_{\max} = \{-20,\, 1,\, 0\}, \quad
\vec{R}_{\min} = \{-200,\, 0,\,-10\}.
\]

The primary objective of the Taxi-v3 experiments is to analyze, in greater depth, the effect introduced in \fref{sec:chickenbanana}, namely the role of \textit{a priori} information about reward magnitudes in dynamic reward weighting. Specifically, we investigate how different configurations of the reward boundaries $(\vec{R}_{\min}, \vec{R}_{\max})$ influence DyLam’s learning behavior, stability, and convergence. By explicitly varying these bounds, we aim to demonstrate their impact on the prioritization of reward components and on the agent’s ability to efficiently learn the passenger delivery task.

In this setting, performance is primarily assessed by the agent’s efficiency in completing the passenger drop-off task. This allows us to isolate how changes in reward normalization and scaling affect task completion without altering the environment dynamics.

\subsubsection{VSS-v0}

The VSS-v0 environment is a continuous, physics-based robotic control task inspired by Very Small Size Soccer (VSS) scenarios~\cite{rsoccer}. In this environment, an agent controls a mobile robot operating on a planar field, with the objective of interacting with a ball and scoring goals against an opponent. An overview of the simulated field and task structure is illustrated in \fref{fig:envs/vss}.

\begin{figure}[ht]
    \centering
    \caption{VSS-v0 environment. A mobile robot interacts with a ball in a planar field, aiming to score goals while minimizing control effort.}
    \includegraphics[width=.6\textwidth]{images/environments/vss.png}
    \par\medskip\ABNTEXfontereduzida\selectfont\textbf{Source:} Author
    \label{fig:envs/vss}
\end{figure}

To facilitate the application of DyLam, the reward function is decomposed into three components that reflect the hierarchical structure of the task:
\begin{enumerate}
    \item \textbf{Move to ball}: a shaping reward encouraging the agent to approach the ball.
    \item \textbf{Ball to goal}: a reward associated with directing the ball toward the opponent’s goal and successfully scoring.
    \item \textbf{Energy}: a penalty proportional to the magnitude of the control actions, discouraging inefficient or excessive motion.
\end{enumerate}

The reward bounds for this environment are defined as
\[
\vec{R}_{\max} = \{150, 40, -100\}, \quad
\vec{R}_{\min} = \{0, 0, -300\}.
\]

The VSS-v0 environment serves as the \textit{motivating benchmark} for the development of DyLam. In preliminary experiments, this task revealed a strong sensitivity to manually selected reward weights, with small variations leading to significantly different behaviors and learning outcomes.

The objective of the VSS-v0 experiments is therefore to demonstrate DyLam’s ability to automatically adapt reward prioritization in a setting where fixed weights are difficult to tune and task objectives are naturally hierarchical. By evaluating performance in terms of goal-scoring rate, we assess whether DyLam can robustly transition from intermediate behaviors (approaching the ball) to task-completion behaviors (scoring goals) without manual intervention in reward design.

Together, Taxi-v3 and VSS-v0 provide complementary perspectives on DyLam’s learning dynamics: the former emphasizes sensitivity to reward scaling and normalization, while the latter highlights the necessity of adaptive weighting in complex robotic control tasks. These environments enable a focused evaluation of how dynamic reward weighting influences convergence speed, stability, and task-relevant performance.


\begin{sidewaystable}[ht]
\centering
\caption{Summary of learning environments used for evaluating DyLam. Pareto-oriented environments focus on trade-off discovery and Pareto front approximation, while learning-dynamics-oriented environments emphasize convergence behavior and task efficiency.}
\label{tab:env_summary}
\begin{tabular}{lcccc}
\toprule
\textbf{Environment} &
\textbf{State/Action Space} &
\textbf{Reward Components} &
\textbf{Performance Assessment} \\
\midrule

MO-HalfCheetah &
Continuous/Continuous &
Forward velocity, Control cost &
Pareto front/Final Position\\

MO-Minecart &
Continuous/Discrete &
Resource collection ($2$ objectives), Fuel &
Pareto front\\

Taxi-v3 &
Discrete/Discrete &
Energy, Passenger drop-off, Illegal action &
Passenger drop-off \\

VSS-v0 &
Continuous/Continuous &
Move to ball, Ball to goal, Energy &
Goal-scoring \\

\bottomrule
\end{tabular}
\end{sidewaystable}

To facilitate comparison across the different experimental setups, we summarize the key characteristics of all environments considered in this work. The environments differ substantially in terms of state and action spaces, reward decomposition, and evaluation criteria, reflecting the complementary roles they play in assessing DyLam. Pareto-oriented environments emphasize trade-off discovery and coverage of the Pareto front, whereas learning-dynamics-oriented environments focus on convergence behavior, reward prioritization, and sample efficiency under practical constraints. \tref{tab:env_summary} provides a unified overview of these properties.
