\chapter{Conclusion}
\label{chap:conclusion}

This work investigated how structured reward decomposition and adaptive reward weighting can be leveraged to improve learning stability and performance in reinforcement learning environments with known, yet conflicting, reward components. Motivated by the limitations of static reward aggregation and manual weight tuning, we studied two complementary approaches: UDC, which decomposes the action-value function into component-wise critics, and DyLam, which dynamically adapts the relative importance of reward components throughout training.

Across a diverse set of environments—ranging from discrete tabular domains to continuous-control and robotic scenarios—our experiments highlighted both the strengths and limitations of decomposition-based methods. UDC consistently behaved as expected in discrete environments with well-aligned objectives, often matching or surpassing traditional baselines and confirming the soundness of its theoretical motivation. However, its performance proved sensitive to learning dynamics and action-space continuity, revealing practical gaps when scaling beyond controlled settings.

In contrast, DyLam demonstrated robust and consistent performance across all evaluated environments, particularly in scenarios characterized by strong or stage-dependent reward conflicts. Rather than seeking multi-objective optimality in the Pareto sense, DyLam reshaped the learning process itself by dynamically prioritizing reward components over time. This behavior led to the emergence of implicit curricula, enabling agents to acquire task-level objectives after learning simpler, \emph{in exploration terms}, sub-tasks. The empirical results in HalfCheetah, Taxi, and especially VSS underscore DyLam’s ability to stabilize learning and reduce reliance on manual reward engineering.

Together, these findings suggest that adaptive reward weighting constitutes a practical and effective mechanism for orchestrating learning progress in complex reinforcement learning problems. The remainder of this chapter summarizes the objectives achieved by this work and outlines promising directions for future research.

\section{Reached Objectives}
\label{sec:reached_objectives}

This dissertation set out to investigate whether adaptive reward weighting could alleviate the limitations of static scalarization and manual tuning in reinforcement learning problems with known, conflicting reward components. The objectives stated in the introduction are revisited below, together with a summary of how they were addressed.

\begin{itemize}
    \item \textbf{Comprehensive review of multi- and sub-objective learning frameworks.}  
    A broad literature review was conducted covering classical multi-objective reinforcement learning, reward decomposition methods, scalarization strategies, and recent adaptive-weight approaches. This review clarified the conceptual distinctions between Pareto-oriented methods and learning-dynamics-oriented formulations, positioning DyLam as a framework inspired by MORL but not restricted to multi-objective optimality. The analysis provided the theoretical grounding necessary to motivate both UDC and DyLam, and to identify open challenges related to reward imbalance and learning instability.

    \item \textbf{Extension of DyLam to discrete state and action spaces.}  
    Originally conceived for continuous-control settings, DyLam was successfully extended to discrete environments by integrating it with tabular and value-based learning algorithms. Experiments in Taxi and Chicken–Banana demonstrated that the method remains well-defined and effective in discrete domains, preserving its adaptive behavior without relying on function approximation or continuous action policies.

    \item \textbf{Demonstration of effective exploration of the reward-weight space.}  
    Through controlled experiments in both Pareto-oriented and learning-dynamics-oriented environments, DyLam was shown to actively explore the reward-weight space by adjusting the $\lambda$-weights in response to component-wise learning progress. Visualizations of weight trajectories and cumulative component rewards confirmed that DyLam does not collapse to a fixed scalarization, but instead continuously reshapes the optimization landscape in a principled and interpretable manner.

    \item \textbf{Adaptive prioritization of under-optimized objectives.}  
    Across all evaluated environments, DyLam consistently prioritized reward components that were lagging behind their expected bounds, while down-weighting those that had already saturated. This mechanism enabled more efficient use of interaction budgets and led to faster convergence on task-relevant performance metrics. The effect was particularly pronounced in environments with stage-dependent or mutually conflicting objectives, such as HalfCheetah and VSS.
\end{itemize}

\section{Future Directions}
\label{sec:future}

Although the results presented in this work demonstrate the effectiveness of DyLam across a range of environments, several promising research directions remain open and constitute a coherent agenda for future investigation.

\begin{itemize}
    \item \textbf{Real-world robotic applications.}  
    Given its strong performance in simulated robotic environments, DyLam is a promising candidate for real-world deployment, where objectives such as energy efficiency, safety, and task performance must be balanced under limited interaction budgets. This research direction can be decomposed into four milestones suitable for a publication: (i) an analysis of the sim-to-real gap in robotic soccer environments such as rSoccer~\cite{rsoccer}; (ii) benchmarking DyLam against state-of-the-art methods in simulation; (iii) investigating DyLam’s ability to reduce the sim-to-real gap with a focus on energy efficiency; and (iv) experimental validation on real rSoccer robotic platforms~\cite{rsoccer}.

    \item \textbf{Extension to partially observable and multi-agent settings.}  
    Extending DyLam to partially observable environments and multi-agent systems introduces additional sources of non-stationarity and coordination complexity. Investigating whether adaptive reward prioritization can mitigate these challenges represents a natural next step. This research direction can be organized into three milestones toward publication: (i) implementation and evaluation of DyLam in partially observable settings, with a focus on non-stationarity arising from partial information; (ii) extension of DyLam to multi-agent environments; and (iii) a comparative analysis of convergence properties and stability in multi-agent scenarios, leveraging established benchmarks such as PettingZoo~\cite{terry2021pettingzoo}.
    
    % \item \textbf{Automatic estimation of reward bounds.}  
    % DyLam currently relies on user-specified reward bounds, namely $\vec{R}_{\max}$ and $\vec{R}_{\min}$. Future work may focus on learning or adapting these bounds online, thereby reducing dependence on domain expertise and further automating reward specification. This research direction can be structured into two main milestones with the goal of producing a standalone publication: (i) a comprehensive literature review on online reward normalization and bound estimation methods; and (ii) the design and evaluation of algorithms for online adaptation of $\vec{R}_{\max}/\vec{R}_{\min}$ within the DyLam framework.

    \item \textbf{Integration with Pareto-front approximation methods.}  
    Although DyLam is not explicitly designed as a Multi-Objective Reinforcement Learning (MORL) method, its adaptive reward-weighting mechanism is conceptually complementary to Pareto-front approximation approaches. Integrating DyLam with MORL techniques may lead to hybrid methods that combine dynamic learning curricula with explicit trade-off exploration among objectives~\cite{pgmorl,alegre2023sample,ropke-ipro}. This scope can be divided into three milestones aimed at a publishable contribution: (i) an in-depth study of MORL-based integration strategies, including methods such as PGMORL and CAPQL~\cite{pgmorl,capql}; (ii) modifications to DyLam’s weighting mechanism to explicitly encourage trade-off exploration; and (iii) extensive benchmarking against state-of-the-art MORL algorithms.

\end{itemize}

Although an extended abstract introducing DyLam has already been published~\cite{machado2025dylam}, we plan to submit a comprehensive journal article that presents the full theoretical formulation, algorithmic details, and extensive empirical evaluation of the DyLam framework. Potential target venues for this submission include well-established journals such as \emph{IEEE Transactions on Pattern Analysis and Machine Intelligence}, \emph{Neurocomputing}, and the \emph{Journal of Machine Learning Research}.

Furthermore, Table~\ref{tab:thesis_timeline} presents a 12-month research roadmap for the completion of the Doctorate degree. The proposed timeline emphasizes the writing phase of each planned scientific contribution and aligns estimated submission periods with high-impact conferences, including the Conference on Robot Learning (CoRL), the Annual Conference on Neural Information Processing Systems (NeurIPS), the AAAI Conference on Artificial Intelligence (AAAI), and the International Conference on Autonomous Agents and Multiagent Systems (AAMAS).


\begin{table}[ht]
    \centering
    \caption{12-Month Research Plan and Thesis Completion Roadmap.}
    \label{tab:thesis_timeline}
\begin{tabular}{|c|c|c|c|c|c|c|c|c|c|c|c|c|}
\hline
Project\textbackslash{}Month      & \textbf{1}               & \textbf{2}               & \textbf{3}               & \textbf{4}               & \textbf{5}               & \textbf{6}               & \textbf{7}               & \textbf{8}                                    & \textbf{9}                                    & \textbf{10}                                   & \textbf{11}              & \textbf{12}              \\ \hline
\textbf{Real-world Robotics}      & \cellcolor[HTML]{000000} & \cellcolor[HTML]{000000} & \cellcolor[HTML]{000000} & \cellcolor[HTML]{FFFFFF} &                          &                          &                          &                                               &                                               &                                               &                          &                          \\ \hline
\textbf{POMDP \& Multi-agent}     &                          &                          & \cellcolor[HTML]{000000} & \cellcolor[HTML]{000000} & \cellcolor[HTML]{000000} &                          &                          &                                               &                                               &                                               &                          &                          \\ \hline
\textbf{DyLam Core}               & \cellcolor[HTML]{000000} & \cellcolor[HTML]{000000} & \cellcolor[HTML]{FFFFFF} & \cellcolor[HTML]{000000} & \cellcolor[HTML]{000000} & \cellcolor[HTML]{000000} & \cellcolor[HTML]{000000} & \cellcolor[HTML]{000000}                      &                                               &                                               &                          &                          \\ \hline
\textbf{Pareto-front Integration} & \multicolumn{1}{l|}{}    & \multicolumn{1}{l|}{}    & \multicolumn{1}{l|}{}    & \multicolumn{1}{l|}{}    & \multicolumn{1}{l|}{}    & \multicolumn{1}{l|}{}    & \multicolumn{1}{l|}{}    & \multicolumn{1}{l|}{\cellcolor[HTML]{000000}} & \multicolumn{1}{l|}{\cellcolor[HTML]{000000}} & \multicolumn{1}{l|}{\cellcolor[HTML]{000000}} & \multicolumn{1}{l|}{}    & \multicolumn{1}{l|}{}    \\ \hline
\textbf{Dissertation Compilation} &                          &                          &                          & \cellcolor[HTML]{FFFFFF} & \cellcolor[HTML]{FFFFFF} & \cellcolor[HTML]{000000} & \cellcolor[HTML]{000000} &                                               &                                               &                                               & \cellcolor[HTML]{000000} & \cellcolor[HTML]{000000} \\ \hline
\end{tabular}
\end{table}